% $Header: /cvsroot/latex-beamer/latex-beamer/solutions/conference-talks/conference-ornate-20min.en.tex,v 1.6 2004/10/07 20:53:08 tantau Exp $

%\documentclass[handout]{beamer}
\documentclass[12 pt]{beamer}
%\documentclass{beamer}f
\setbeamertemplate{navigation symbols}{}

%\usepackage{pstricks}
\usepackage[mathcal]{euscript}

\usepackage{graphicx}
\usepackage{diagrams}
\usepackage{tikz}
\usetikzlibrary{calc}

\input{rflatexmacs}
\newcommand{\SSS}{\text{\emphslb{S}}}
\newcommand{\bL}{\alg L}
\newcommand{\bB}{\alg B}



%\definecolor{MyDarkGreen}{rgb}{0.2,0.7,0.7}
\definecolor{MyDarkGreen}{rgb}{0.2,0.6,0.9}
\definecolor{MyDarkBlue}{rgb}{0.2,0.2,0.7}
%\definecolor{exercisecolor}{rgb}{0.0,0.8,0.8}
\definecolor{exercisecolor}{rgb}{0.3,0.2,0.8}
\newcommand{\emphcyan}[1]{\textcolor{MyDarkBlue}{\textbf{#1}}}
% \newcommand{\emphcyan}[1]{\textcolor{cyan}{#1}}

%\newbox\bigstrutbox
%\setbox\bigstrutbox=\hbox{\vrule height12pt depth4pt width0pt}
%\def\bigstrut{\relax\ifmmode\copy\bigstructbox\else\unhcopy\bigstructbox\fi}

% This file is a solution template for:

% - Talk at a conference/colloquium.
% - Talk length is about 20min.
% - Style is ornate.



% Copyright 2004 by Till Tantau <tantau@users.sourceforge.net>.
%
% In principle, this file can be redistributed and/or modified under
% the terms of the GNU Public License, version 2.
%
% However, this file is supposed to be a template to be modified
% for your own needs. For this reason, if you use this file as a
% template and not specifically distribute it as part of a another
% package/program, I grant the extra permission to freely copy and
% modify this file as you see fit and even to delete this copyright
% notice. 


\mode<presentation>
{
  %\usetheme{Szeged}
  %\usetheme{default}
  \usetheme{Madrid}
  \usecolortheme{spruce}
  \setbeamercolor{block title}{use=structure,fg=white,bg=cyan!60!black}
  % or ...

  %\setbeamercovered{transparent}
  % or whatever (possibly just delete it)
}

\usepackage[OT2,T1]{fontenc}
\usepackage[russian,USenglish]{babel}
%\usepackage[english]{babel}
% or whatever

\usepackage[latin1]{inputenc}
% or whatever

\usepackage{times}
\usepackage[T1]{fontenc}
% Or whatever. Note that the encoding and the font should match. If T1
% does not look nice, try deleting the line with the fontenc.


\title[Representing Finite Lattices]
{Representing Finite Lattices as \\Congruence Lattices}

\author{William DeMeo, Ralph Freese, Peter Jipsen}
%\author{Ralph Freese}
%\institute{Hawaii}
% - Give the names in the same order as the appear in the paper.
% - Use the \inst{?} command only if the authors have different
%   affiliation.

%\institute[University of Hawaii] % (optional, but mostly needed)
%{
  %\inst{1}%
  %Department of Computer Science\\
  %University of Somewhere
  %\and
  %\inst{2}%
  %Department of Theoretical Philosophy\\
  %University of Elsewhere}
% - Use the \inst command only if there are several affiliations.
% - Keep it simple, no one is interested in your street address.


\date[Aug 14--18, 2017]% (optional, should be abbreviation of conference name)
{
\vspace{-12 mm}
\begin{align*}
&\texttt{http://math.hawaii.edu/$\sim$ralph/}\\
&\texttt{http://uacalc.org/}\\
&\texttt{https://github.com/UACalc/}
\end{align*}
}


%{\newline The new results of this talk are joint work with Paolo Lipparini.}

%\newline
%\texttt{\small http://www.uacalc.org/}}

% - Either use conference name or its abbreviation.
% - Not really informative to the audience, more for people (including
%   yourself) who are reading the slides online

%\subject{Theoretical Computer Science}
% This is only inserted into the PDF information catalog. Can be left
% out. 



% If you have a file called "university-logo-filename.xxx", where xxx
% is a graphic format that can be processed by latex or pdflatex,
% resp., then you can add a logo as follows:

% \pgfdeclareimage[height=0.5cm]{university-logo}{university-logo-filename}
% \logo{\pgfuseimage{university-logo}}



% Delete this, if you do not want the table of contents to pop up at
% the beginning of each subsection:

%\AtBeginSubsection[]
%{
%  \begin{frame}<beamer>
%    \frametitle{Outline}
%    \tableofcontents[currentsection,currentsubsection]
%  \end{frame}
%}


% If you wish to uncover everything in a step-wise fashion, uncomment
% the following command: 

%\beamerdefaultoverlayspecification{<+->}





\begin{document}

\thicklines

\begin{frame}
  \titlepage
   
  
  %\vspace{0.2in}
  \centerline{BLAST, Vanderbilt University, Aug 14--18, 2017}


\end{frame}


\begin{frame}[t]
  \frametitle{The Problem}

\begin{theorem}[Gr\"atzer-Schmidt]
Every algebraic (so every finite) lattice is isomorphic to 
$\Con(\alg A)$ for some (unary) algebra 
$\op{\alg A}$.
\end{theorem}

\uncover<2->{
\begin{problem} 
Is every finite $\alg L$ isomorphic to $\Con (\alg A)$ for some
\textcolor{purple}{\textbf{finite}} $\alg A$?
\end{problem}
}

\uncover<3->{
Since
$\Con(\alg A) = \Con\langle A, \op{Pol}_1(\alg A) \rangle$,
\textcolor{purple}{\textbf{we assume all algebras are unary.}}
}


\end{frame}

\begin{frame}[t]
  \frametitle{Properties}


Possible representation properties for a finite lattice $\alg L$:

\begin{enumerate}
\item[(P1)]
$\alg L$ is isomorphic to the congruence
lattice of some finite algebra $\langle A,F \rangle$.
\item[(P2)]\uncover<2->{
$\alg L$ is isomorphic to the congruence
lattice of some finite algebra $\langle A,F \rangle$
where the all nonconstant operations are permutations.}
\item[(P3)]\uncover<3->{
$\alg L$ is isomorphic to the congruence
lattice of some finite algebra $\langle A,F \rangle$
where the nonconstant operations generate a transitive
permutation group.}
\item[(P4)]\uncover<4->{
$\alg L$ is
isomorphic to an interval in the lattice of subgroups of a finite
group.}
\end{enumerate}

\centerline{\uncover<6->{(P4) $\Leftrightarrow$} 
\uncover<5->{(P3)
$\Rightarrow$ (P2)
$\Rightarrow$ (P1)}
}

\end{frame}

\begin{frame}[t]
  \frametitle{(P3) $\Leftrightarrow$ (P4)}
Let $\alg H$ be a subgroup of $\alg G$. Let
\[
A = \{aH : a \in G\}\quad\text{(left cosets)}
\]  
\uncover<2->{
Make an algebra $\alg A$ by adding operations 
\[
g : aH \mapsto gaH \qquad\qquad\text{(left multiplication)}
\]
}
\uncover<3->{
Then $\Con \alg A \iso [H,G]$, the interval in the subgroup lattice.
}

\end{frame}


\begin{frame}[t]
  \frametitle{P\'alfy-Pudl\'ak}
  
\begin{theorem}[1980]
\textup{(P1)} holds for 
\textcolor{purple}{all}
lattices iff \textup{(P4)} holds for
\textcolor{purple}{all} lattices.
\end{theorem}

\uncover<2->{
The \emphcyan{size} of a representation 
$\alg L \iso \Con(\alg A)$ is $|A|$. For $\alg H \le \alg G$ the size
in (P4) is $[G : H]$, the number of left $H$-cosets of $\alg G$.
}

\medskip

\uncover<3->{
\textbf{Example.} The minimum size for $\alg L_6$ is 6:

\medskip

\begin{tabular}{ccc}
$\alg L_6$&
\begin{minipage}{0.07\textwidth}
\begin{tikzpicture}
    [scale=0.6, e/.style={circle,draw,inner sep=0pt,minimum size=4pt}]
\node(5) at (0,1)[e]{};
\node(4) at (-0.5,0.33)[e]{};
\node(3) at (0.5,0.33)[e]{};
\node(2) at (-0.5,-0.33)[e]{};
\node(1) at (0.5,-0.33)[e]{};
\node(0) at (0,-1)[e]{};
\node at (0,1.3){};
\draw(4)--(5);
\draw(3)--(5);
\draw(2)--(4);
\draw(1)--(3);
\draw(0)--(1);
\draw(0)--(2);
\end{tikzpicture}
\end{minipage}
}
\uncover<4->{
&
$\setlength{\arraycolsep}{1pt}\begin{array}{c|cccccc}
    \alg B_6& 0& 1& 2& 3& 4& 5\\\hline
   f(x)& 2& 2& 1& 5& 5& 4\\
   g(x)& 3& 4& 4& 0& 1& 1\\
   h(x)& 4& 5& 3& 4& 5& 3\end{array}$
\end{tabular}
}

\medskip
\uncover<5->{
P\'alfy and Aschbacher have found groups $\alg H \le \alg G$
representing this lattice. 
But P\'alfy's example has $\alg G = \alg A_{11}$ and $|H| = 55$, 
so the size is $9! = 362880$.
}
\end{frame}

\begin{frame}[t]
  \frametitle{Moral}

\emphcyan{Moral:} Finding a representation with groups, (P4),
may be much harder (and much bigger) than finding 
a (P1) representation.


\end{frame}

\begin{frame}[t]
  \frametitle{New representable lattices from old}

\begin{itemize} 
\item all distributive lattices
\uncover<2->{
\item lattice duals (Hans Kurzweil, 1985, and R. Netter, 1986)}
\uncover<3->{\item interval sublattices}
\uncover<4->{\item  direct products (Ji\v{r}\'i T\r{u}ma,  1986)}
\uncover<5->{\item  ordinal sums
  (Ralph McKenzie, 1984; John Snow, 2000)}
\uncover<6->{\item  parallel sums (John Snow, 2000)}
\uncover<7->{\item sublattices of representable lattices 
obtained as a union of a filter and an ideal
  (John Snow, 2000)}
\uncover<8->{\item
overalgebras (DeMeo, 2013)}
\end{itemize}

\end{frame}

\begin{frame}[t]
  \frametitle{P\'alfy-Pudl\'ak Conditions}

\begin{enumerate}
\item[(A)]
$\alg L$ is simple.
\uncover<2->{\item[(B)]
For each $x \ne 0$ in $L$, there are elements $y$ and $z$
such that $x \join y = x \join z = 1$ and $y\meet z = 0$.}
\uncover<3->{\item[(C)] $|L|\ne 2$ and each element of $L$ that is not an 
atom or $0$ contains at least four atoms.}
\end{enumerate}

\uncover<4->{
\begin{theorem}
\begin{itemize}
\item
If\/ $\alg L$ satisfies \textup{(A)} and \textup{(B)} then $\alg L$ 
satisfies \textup{(P1)} $\Rightarrow$ \textup{(P2)}.}
\uncover<5->{\item
If\/ $\alg L$ satisfies \textup{(A)}, \textup{(B)} and \textup{(C)} 
then $\alg L$ satisfies \textup{(P1)} $\Rightarrow$ \textup{(P3)}.}
\end{itemize}
\end{theorem}


\end{frame}

\begin{frame}[t]
  \frametitle{McKenzie's variants}
  
\begin{enumerate}
\item[(B${}'$)]
If $\varphi: L \to L$ is any meet-preserving map such 
that $\varphi(x) > x$ for
$x\ne 1$, then $\varphi(x) = 1$ for all~$x$.
\uncover<2->{\item[(B$''$)]
The coatoms of $L$ meet to $0$.}
\end{enumerate}

\uncover<3->{
\[
\text{(B)} \implies \text{(B$''$)}  \implies \text{(B$'$)}.
\]
}


\uncover<4->{\begin{theorem}
\begin{itemize}
\item
If\/ $\alg L$ satisfies \textup{(A)} and \textup{(B${}'$)}  \textup{(or (B${}''$))}
then a minimal representation of\/ $\alg L$ witnesses that
$\alg L$ satisfies \textup{(P2)}. So,}
%\uncover<5->{\item
%If\/ $\alg L$ satisfies \textup{(A)} and \textup{(B${}''$)} 
%then $\alg L$ satisfies \textup{(P2)}.}
\uncover<5->{\item If \textup{(A)} and \textup{(B${}'$)} hold and\/ 
$\alg L \cong \la A,F \ra$ 
is minimal,  then $F$ consists of permutations
and constants. }
\end{itemize}
\end{theorem}

\end{frame}

% to get a weakening of (C) we study intransitive group action.


\begin{frame}[t]
  \frametitle{Representations by instransitive groups}

Suppose $\alg A = \la A,G\ra$ is a $G$-set and 
let $\alg A_i = \la A_i, G\ra$, $i < k$,
be the minimal subalgebras of $\alg A$; i.e.~each set $A_i$ is an 
orbit, or
one-generated subuniverse, of $\alg A$. 

\uncover<2->{
Define congruences on $A$ by the 
partitions}
\uncover<2->{
\begin{alignat*}{2}
\tau &= |A_0|A_1|\cdots|A_{k-1}| &&\quad\text{ (the blocks are the orbits)} \\
\tau_i &= |A_i| &&\quad\text{ (at most one nontrivial block) }\\
\gamma_i &= |A_i|A - A_i| &&\quad\text{ (exactly two blocks unless
$A_i = A$) } 
\end{alignat*}
}
\uncover<3->{
We call $\tau$ the \emphcyan{intransitivity congruence}; 
%$G$ acts transitively if and only if $\tau = 1$. 
}

\uncover<4->{
\begin{theorem}
Let $\theta \in \Con(\alg A)$, where 
$\alg A = \la A,G\ra$ and $G$ is a group. Then
\end{theorem}
}
\end{frame}

\begin{frame}[t]
  %\frametitle{Representations by intransitive groups}

%\begin{theorem}
%Let $\alg A = \la A,G\ra$ and let $\theta \in \Con(\alg A)$.
\vspace{-3 mm}

\[
\tau = |A_0|A_1|\cdots|A_{k-1}| \quad\text{ (the blocks are the orbits) }
\]
\vspace{-5 mm}
\begin{enumerate}
\item\label{item1}
$G$ acts transitively if and only if $\tau = 1_{\alg A}$.
\uncover<2->{\item \label{item2}
The interval $[\tau,1_\alg A]$ is isomorphic to $\operatorname{Eq}(k)$.}
\uncover<3->{\item \label{item3}
The interval $[ 0_\alg A,\tau]$ is isomorphic 
to $\prod_{i=0}^{k-1} \Con(\alg A_i)$.}

\only<4>{\textcolor{MyDarkGreen}{For $\theta_i \in \Con(\alg A_i)$, map
$(\theta_0,\ldots,\theta_{k-1}) \mapsto \theta_0\union \cdots \union \theta_{k-1}$.
}}

\uncover<5->{\item\label{item4}
If, for some $i$, $\theta \ge \Join_{j \ne i} \tau_j$ 
then $\theta \ge \tau$
or $\theta \le \gamma_i$.}

\only<6>{\textcolor{MyDarkGreen}{If $\theta \nleq \gamma_i$, there are $a\in A_i$ and $b \notin A_i$ with
$(a,b) \in \theta$. Since $G$ acts transitively on each orbit, $\tau_i \le \theta$.
So $\tau \le \theta$.}}



\uncover<7->{\item\label{item5}
If $\theta\meet \tau \prec \tau$  then $\theta \le \gamma_i$ 
for some~$i$.}

\only<8>{\textcolor{MyDarkGreen}{Follows from (3) and (4).}}


\uncover<9->{\item\label{item6}
If $k>1$ and $|A_i| = 1$ for all $i$ except $0$ then every coatom of $\Con(\alg A)$
lies above~$\tau$.}
\uncover<10->{\item\label{item7}
If $k>1$ and $[ 0_\alg A,\tau]$ is directly indecomposable 
then every coatom of $\Con(\alg A)$ lies above~$\tau$.}
\uncover<11->{\item\label{item8}
If $k = 2$ and $|A_1| = 1$ then $\tau$ is a coatom and 
everything is comparable with it.}

\only<12>{\textcolor{MyDarkGreen}{From (2) and (6). }}
\uncover<13->{\item\label{item9}
If $\tau$ is a coatom and $[ 0_\alg A,\tau]$ is directly indecomposable
then everything is comparable with it.}

\only<14>{\textcolor{MyDarkGreen}{From (8).}}
\end{enumerate}
%\end{theorem}

\end{frame}

\begin{frame}[t]
  \frametitle{Examples: $\alg L_{14}$}
  
  
\vspace{-1 cm}
  
\begin{center}  
\begin{tikzpicture}
    [scale=1.1, e/.style={circle,draw,inner sep=0pt,minimum size=4pt}]
\node(6) at (0,1)[e]{};
\node(5) at (0.4,0.33)[e]{};
\node(4) at (0,0.33)[e]{};
\node(3) at (-0.8,0)[e]{};
\node(2) at (-0.4,0)[e]{};
\node(1) at (0.2,-0.33)[e]{};
\node(0) at (0,-1)[e]{};
\node at (0,1.3){};
\uncover<2->{
\draw (5) node [right]{$\tau$};
}
\draw(5)--(6);
\draw(4)--(6);
\draw(3)--(6);
\draw(2)--(6);
\draw(1)--(4);
\draw(1)--(5);
\draw(0)--(1);
\draw(0)--(2);
\draw(0)--(3);
\end{tikzpicture}
\end{center}

\vspace{-3 mm}

\begin{example}
\begin{itemize}
\item
$\alg L_{14}$ satisfies (A) and (B${}''$)
so a minimal representation is permutational.
\uncover<3->{
\item
$\alg L_{14} \cong \Con \la A,G\ra$ is not possible if $G$ acts intransitively,
} 
\uncover<4->{
so
\item
if $\Con \la A,F\ra$ is a minimal representation,
then $F$ generates a transitive group.}
\uncover<5->{
\item
Is $\alg L_{14}$ representable? (Yes: as $[H,A_6]$ with $[A_6 : H] =90$)}
\uncover<6->{
\item
Is this a minimum representation? (Don't know)}
\end{itemize}
\end{example}

%by items (1), (2) and (7).

\end{frame}

\begin{frame}[t]
  \frametitle{Examples: $\alg L_{15}$, (the dual of $\alg L_{14}$)}
  
  
\vspace{-1 cm}
  
\begin{center} 
\begin{tikzpicture}
    [scale=1.3, e/.style={circle,draw,inner sep=0pt,minimum size=4pt}]
\node(6) at (0,1)[e]{};
\node(5) at (0.2,0.33)[e]{};
\node(4) at (-0.4,0)[e]{};
\node(3) at (-0.8,0)[e]{};
\node(2) at (0,-0.33)[e]{};
\node(1) at (0.4,-0.33)[e]{};
\node(0) at (0,-1)[e]{};
\node at (0,1.3){};
\uncover<2->{
\draw (5) node [right]{$\tau$};
}
\draw(5)--(6);
\draw(4)--(6);
\draw(3)--(6);
\draw(2)--(5);
\draw(1)--(5);
\draw(0)--(4);
\draw(0)--(3);
\draw(0)--(2);
\draw(0)--(1);
\end{tikzpicture}
\end{center}

\uncover<3->{
\begin{example}
\begin{itemize}
\item
$\alg L_{15} \cong \Con \la \{0,1,2,3\},G\ra$, $G$ the group
generated by the double transposition 
$0 \leftrightarrow 1$, $2 \leftrightarrow 3$.}
\uncover<4->{
\item
$\alg L_{14} \cong \alg L^d_{15}$,
which again proves $\alg L_{14}$ is representable.}

\end{itemize}
\end{example}

%by items (1), (2) and (7).

\end{frame}

\begin{frame}[t]
  \frametitle{Examples: $\alg L_{4}$.}
  
  
\vspace{-1 cm}

\begin{center}
\begin{tikzpicture}
    [scale=1.2, e/.style={circle,draw,inner sep=0pt,minimum size=4pt}]
\node(5) at (0,1)[e]{};
\node(4) at (0.2,0.33)[e]{};
\node(3) at (-0.5,0)[e]{};
\node(2) at (0.4,-0.33)[e]{};
\node(1) at (0,-0.33)[e]{};
\node(0) at (0,-1)[e]{};
\node at (0,1.3){};
\draw (4) node [right]{$\tau$};
\draw(4)--(5);
\draw(3)--(5);
\draw(2)--(4);
\draw(1)--(4);
\draw(0)--(3);
\draw(0)--(2);
\draw(0)--(1);
\end{tikzpicture}
\end{center}

\uncover<2->{
\begin{example}
\begin{itemize}
\item

$\alg L_{4}$ satisfies (B${}''$) but not (A)
so minimal representations need not be permutational.}
%\item 
\only<3>{
In fact
\item
$\alg L_4 \cong \la \{0,1,2,3\}, f, g \ra$, where
\[
\setlength{\arraycolsep}{1pt}\begin{array}{c|cccc}
    \alg B_4& 0& 1& 2& 3\\\hline
   f(x)& 1& 0& 3& 2\\
   g(x)& 0& 0& 2& 2\end{array}
\]
}
\only<4>{
\item
But $\alg L_4$ does have an intransitive representation on 6:
\[
\setlength{\arraycolsep}{1pt}\begin{array}{c|cccccc}
    \alg B'_4& 0& 1& 2& 3& 4& 5\\\hline
   f(x)& 1& 2& 0& 4& 5& 3\\
   g(x)& 0& 2& 1& 3& 5& 4\end{array}
\]
}

\end{itemize}
\end{example}

\end{frame}

\begin{frame}[t]
  \frametitle{Examples: $\alg L_{19}$, a harder example:}

\vspace{-10 mm}

\begin{center}
\begin{tikzpicture}
    [scale=1.2, e/.style={circle,draw,inner sep=0pt,minimum size=4pt}]
\node(6) at (0,1)[e]{};
\node(5) at (-0.5,0.33)[e]{};
\node(4) at (0.5,0.33)[e]{};
\node(3) at (-0.5,-0.33)[e]{};
\node(2) at (0,-0.33)[e]{};
\node(1) at (0.5,-0.33)[e]{};
\node(0) at (0,-1)[e]{};
\node at (0,1.3){};
\draw (5) node [left]{$\tau$};
\draw (4) node [right]{$\theta$};
\draw (1) node [right]{$\psi$};

\draw(5)--(6);
\draw(4)--(6);
\draw(3)--(5);
\draw(2)--(5);
\draw(1)--(4);
\draw(0)--(3);
\draw(0)--(2);
\draw(0)--(1);
\end{tikzpicture}
\end{center}

\uncover<2->{
\begin{lemma}
Let $\alg A = \la A,G\ra$ be a finite algebra, where
$G$ is an intransitive group of permutations on $A$.
Suppose the intransitivity congruence $\tau$ is a coatom.
Then there do not exist congruences
$0_{\alg A} < \psi < \theta$ in $\Con(\alg A)$
with $\theta \meet \tau = 0_{\alg A}$. 
\end{lemma}
}




\end{frame}

\begin{frame}[t]
  \frametitle{Proof}

\begin{lemma}
Let $\alg A = \la A,G\ra$ be a finite algebra, where
$G$ is an intransitive group of permutations on $A$.
Suppose the intransitivity congruence $\tau$ is a coatom.
Then there do not exist congruences
$0_{\alg A} < \psi < \theta$ in $\Con(\alg A)$
with $\theta \meet \tau = 0_{\alg A}$. 
\end{lemma}

\begin{proof}
Since $\tau$ is a coatom, there are exactly two orbits; call
them $B$ and $C$. Since $\theta \meet \tau = 0_{\alg A}$, if
$(x,y) \in \theta$ then $x=y$ or one is in $B$ and the other
is in $C$. So $\theta$ defines a bipartite graph between
$B$ and $C$. Since $G$ acts transitively on both $B$ and $C$,
this graph corresponds to a bijection between $B$ and $C$. 
The same applies to $\psi$. But equivalence relations
corresponding to such graphs cannot be comparable.
\end{proof}

  
\end{frame}

\begin{frame}[t]
  \frametitle{Small Lattices}

\begin{theorem}
All lattices with at most 7 elements can be represented, with the
one possible exception of\/ $\alg L_{10}$:
\vspace{-9mm}
\begin{center}
\begin{tikzpicture}
    [scale=1.2, e/.style={circle,draw,inner sep=0pt,minimum size=4pt}]
\node(6) at (0.33,1)[e]{};
\node(5) at (0,0.33)[e]{};
\node(4) at (0.66,0.33)[e]{};
\node(3) at (-1,0)[e]{};
\node(2) at (-0.33,-0.33)[e]{};
\node(1) at (0.33,-0.33)[e]{};
\node(0) at (0,-1)[e]{};
\node at (0,1.3){};
\uncover<3->{
\draw (4) node [right] {$\theta$};
\draw (5) node [right] {$\tau$};
}
\draw(5)--(6);
\draw(4)--(6);
\draw(3)--(6);
\draw(2)--(5);
\draw(1)--(4);
\draw(1)--(5);
\draw(0)--(1);
\draw(0)--(2);
\draw(0)--(3);
\end{tikzpicture}
\end{center}
\uncover<2->{
If\/ $\alg L_{10} \cong \la A,F\ra$, then $F$ generates a transitive 
group on $A$.
}
\end{theorem}

\uncover<3->{
\begin{proof}
$\alg L_{10}$ satisfies (A) and (B${}''$). By part (5) of the 
intransitivity theorem, it cannot be represented with an 
intransitive group. 
\end{proof}
}

\end{frame}

\begin{frame}[t]
\frametitle{Finding Reps: Methods and Algorithms} 

\begin{itemize}
\uncover<2->{
\item
Closure Method}
\uncover<3->{
\item
Overalgebras}
\uncover<4->{
\item
Ideal-Filter}
\uncover<5->{
\item
Duality}
\uncover<6->{
\item
    Group Methods (GAP)}
\end{itemize}

\end{frame}

\begin{frame}[t]
\frametitle{Closure Method to find a Representation of $\alg L$} 
\begin{itemize}
\uncover<2->{
\item[(1)]
Search through $\operatorname{Eq}(X_k)$, $k = 2,3, \ldots$ finding
sublattices isomorphic to~$\alg L$. }
\uncover<3->{
\item[(2)]
For each sublattice $\alg L \cong \alg L' \le \operatorname{Eq}(X_k)$ found, 
find the unary polymorphs of the members of $L'$; that is,
calculate the set $F$ of all unary operations on $X_k$
which respect all $\theta \in L'$.}
\uncover<4->{
\item[(3)]
For $F$ found in the previous step, test if
$\Con(\langle X_k,F \rangle) = L'$. 
If so then $\alg A = \langle X_k,F \rangle$ is a minimal 
representation. Otherwise continue the search.}


\end{itemize}

\end{frame}

\begin{frame}[t]
\frametitle{Remarks}

\begin{itemize}
%\item
%Of course if $\alg L$ has no representation, this procedure will 
%not converge.
%Moreover, seaching for $\alg L$ in $\operatorname{Eq}(k)$
%becomes quickly prohibitive. 

\item[(a)]
\textbf{Find a small presentation of $\alg L$:}

The procedure can be sped up by first finding a presentation
of $\alg L$ with the minimal number of generators. Besides speeding
up the search in $\operatorname{Eq}(k)$, it is enough in 
calculating the unary polymorphs to respect the generators.

\end{itemize}
\end{frame}

\begin{frame}[t]
\frametitle{Remarks}

\begin{itemize}

\item[(b)]
\textbf{Subdirect Decompositions:}


Subdirect decompositions can be used to speed up finding unary
polymorphs. For  example, if $\theta_0$, $\theta_1 \in L' \le 
\operatorname{Eq}(X_k)$ with $\theta_0 \meet \theta_1 = 0$, 
then $X_k$ is naturally embedded into 
$X_k/\theta_0 \times X_k/\theta_1$. Since the operations in a
direct product are component-wise, this 
cuts the search space of possible unary polymorphs 
from $k^k$ down to $r^r  s^s$, where $r$ and $s$ are the number 
of blocks in $\theta_0$ and~$\theta_1$.

\end{itemize}
\end{frame}


\begin{frame}[t]
\frametitle{Remarks}

\begin{itemize}

\item[(c)]
\textbf{Uniform Equivalence Relations:}

If it can be shown that the algebra of a minimal representation 
of $\alg L$ has a transitive permutation group for its nonconstatant
unary polynomials, then we can restrict our search in 
$\operatorname{Eq}(k)$ to uniform equivalence relations.
Moreover the search for unary polymorphs can be restricted
to permutations. 

\end{itemize}
\end{frame}


\begin{frame}[t]
\frametitle{Remarks}

\begin{itemize}

\item[(d)]
\textbf{Small generating set for the operations:}

Of course if $F' \subseteq F$ is a set of generators for the 
moniod $F$, we can take $\alg A = \langle X_k,F' \rangle$.

\end{itemize}


\end{frame}

\begin{frame}
\frametitle{Nondist., linearly indec., small lattices} 


\setlength{\tabcolsep}{2pt}
\begin{tabular}{ccc}
$\alg L_1$&\quad
\begin{minipage}{0.06\textwidth}
\begin{tikzpicture}
    [scale=0.6, e/.style={circle,draw,inner sep=0pt,minimum size=4pt}]
\node(4) at (0,1)[e]{};
\node(3) at (0.33,0.33)[e]{};%[$1\ 2|3\ 4$]
\node(2) at (-0.5,0.0)[e]{};%$1\ 3|2\ 4$};
\node(1) at (0.33,-0.33)[e]{};%$1|2|3\ 4$};
\node(0) at (0,-1)[e]{};
\node at (0,1.3){};
\draw(3)--(4);
\draw(2)--(4);
\draw(1)--(3);
\draw(0)--(1);
\draw(0)--(2);
\end{tikzpicture}
\end{minipage}
&
$\setlength{\arraycolsep}{1pt}
\begin{array}{c|cccc}
    \alg B_1& 0& 1& 2& 3\\\hline
   f(x)& 1& 0& 3& 2\\
   g(x)& 1& 0& 1& 0\end{array}$
\end{tabular}

\ 

\begin{tabular}{ccc}
$\bL_2$&\quad
\begin{minipage}{0.07\textwidth}
\begin{tikzpicture}
    [scale=0.6, e/.style={circle,draw,inner sep=0pt,minimum size=4pt}]
\node(4) at (-0.0,1.0)[e]{};
\node(3) at (0.5,0.0)[e]{};
\node(2) at (0.0,0.0)[e]{};
\node(1) at (-0.5,0.0)[e]{};
\node(0) at (-0.0,-1.0)[e]{};
\node at (0,1.3){};
\draw(3)--(4);
\draw(2)--(4);
\draw(1)--(4);
\draw(0)--(1);
\draw(0)--(2);
\draw(0)--(3);
\end{tikzpicture}
\end{minipage}
&
$\setlength{\arraycolsep}{1pt}\begin{array}{c|cccc}
    \bB_2& 0& 1& 2\\\hline
   f(x)& 0& 1& 2\end{array}$
\end{tabular}



\begin{tabular}{ccc}
$\bL_3$&\quad
\begin{minipage}{0.07\textwidth}
\begin{tikzpicture}
    [scale=0.6, e/.style={circle,draw,inner sep=0pt,minimum size=4pt}]
\node(5) at (0,1)[e]{};
\node(4) at (0.5,0.33)[e]{};
\node(3) at (-0.5,0.0)[e]{};
\node(2) at (0,0)[e]{};
\node(1) at (0.5,-0.33)[e]{};
\node(0) at (0,-1)[e]{};
\node at (0,1.3){};
\draw(4)--(5);
\draw(3)--(5);
\draw(2)--(5);
\draw(1)--(4);
\draw(0)--(1);
\draw(0)--(2);
\draw(0)--(3);
\end{tikzpicture}
\end{minipage}
&
$\setlength{\arraycolsep}{1pt}\begin{array}{c|ccccccc}
           \bB_3& 0& 1& 2& 3& 4& 5& 6\\\hline
   f(x)& 0& 1& 2& 1& 2& 1& 0\\
   g(x)& 0& 3& 4& 3& 4& 3& 0\\
   h(x)& 6& 5& 2& 5& 2& 5& 6\\
   k(x)& 0& 1& 2& 0& 0& 2& 2\end{array}$
\end{tabular}

\qquad\qquad \textcolor{MyDarkGreen}{Method: overalgebras}


\end{frame}

\begin{frame}
%\frametitle{Small Lattices}

\begin{tabular}{ccc}
$\bL_4$&
\begin{minipage}{0.07\textwidth}
\begin{tikzpicture}
    [scale=0.6, e/.style={circle,draw,inner sep=0pt,minimum size=4pt}]
\node(5) at (0,1)[e]{};
\node(4) at (0.2,0.33)[e]{};
\node(3) at (-0.5,0)[e]{};
\node(2) at (0.4,-0.33)[e]{};
\node(1) at (0,-0.33)[e]{};
\node(0) at (0,-1)[e]{};
\node at (0,1.3){};
\draw(4)--(5);
\draw(3)--(5);
\draw(2)--(4);
\draw(1)--(4);
\draw(0)--(3);
\draw(0)--(2);
\draw(0)--(1);
\end{tikzpicture}
\end{minipage}
&
$\setlength{\arraycolsep}{1pt}\begin{array}{c|cccc}
    \bB_4& 0& 1& 2& 3\\\hline
   f(x)& 1& 0& 3& 2\\
   g(x)& 0& 0& 2& 2\end{array}$
\end{tabular}

\medskip

\begin{tabular}{ccc}
$\bL_5$&
\begin{minipage}{0.07\textwidth}
\begin{tikzpicture}
    [scale=0.6, e/.style={circle,draw,inner sep=0pt,minimum size=4pt}]
\node(5) at (0,1)[e]{};
\node(4) at (0,0.33)[e]{};
\node(3) at (0.4,0.33)[e]{};
\node(2) at (-0.5,0)[e]{};
\node(1) at (0.2,-0.33)[e]{};
\node(0) at (0,-1)[e]{};
\node at (0,1.3){};
\draw(4)--(5);
\draw(3)--(5);
\draw(2)--(5);
\draw(1)--(3);
\draw(1)--(4);
\draw(0)--(1);
\draw(0)--(2);
\end{tikzpicture}
\end{minipage}
&
$\setlength{\arraycolsep}{1pt}\begin{array}{c|cccccccccccc}
           \bB_5& 0& 1& 2& 3& 4& 5& 6& 7& 8& 9 & 10& 11\\\hline
    f(x)& 1&  2&   3& 4& 5& 0& 7& 8& 9& 10& 11& 6\\
   g(x)& 6& 11& 10& 9& 8& 7& 0& 5& 4& 3&   2&  1\\
   h(x)& 0& 0& 0& 6& 0& 0& 0& 0& 6& 0& 0& 0\end{array}$
\end{tabular}

\medskip

\begin{tabular}{ccc}
$\bL_6$&
\begin{minipage}{0.07\textwidth}
\begin{tikzpicture}
    [scale=0.6, e/.style={circle,draw,inner sep=0pt,minimum size=4pt}]
\node(5) at (0,1)[e]{};
\node(4) at (-0.5,0.33)[e]{};
\node(3) at (0.5,0.33)[e]{};
\node(2) at (-0.5,-0.33)[e]{};
\node(1) at (0.5,-0.33)[e]{};
\node(0) at (0,-1)[e]{};
\node at (0,1.3){};
\draw(4)--(5);
\draw(3)--(5);
\draw(2)--(4);
\draw(1)--(3);
\draw(0)--(1);
\draw(0)--(2);
\end{tikzpicture}
\end{minipage}
&
$\setlength{\arraycolsep}{1pt}\begin{array}{c|cccccc}
    \bB_6& 0& 1& 2& 3& 4& 5\\\hline
   f(x)& 2& 2& 1& 5& 5& 4\\
   g(x)& 3& 4& 4& 0& 1& 1\\
   h(x)& 4& 5& 3& 4& 5& 3\end{array}$
\end{tabular}

\end{frame}

\begin{frame}
%\frametitle{Small Lattices}

\begin{tabular}{ccc}
$\bL_7$&
\begin{minipage}{0.07\textwidth}
\begin{tikzpicture}
    [scale=0.6, e/.style={circle,draw,inner sep=0pt,minimum size=4pt}]
\node(5) at (0,1)[e]{};
\node(4) at (0.5,0.5)[e]{};
\node(3) at (-0.5,0)[e]{};
\node(2) at (0.5,0)[e]{};
\node(1) at (0.5,-0.5)[e]{};
\node(0) at (0,-1)[e]{};
\node at (0,1.3){};
\draw(4)--(5);
\draw(3)--(5);
\draw(2)--(4);
\draw(1)--(2);
\draw(0)--(1);
\draw(0)--(3);
\end{tikzpicture}
\end{minipage}
&
$\setlength{\arraycolsep}{1pt}\begin{array}{c|cccccc}
    \bB_7& 0& 1& 2& 3& 4& 5\\\hline
   f(x)& 1& 0& 0& 4& 3& 3\\
   g(x)& 4& 5& 5& 1& 2& 2\\
   h(x)& 3& 3& 4& 3& 3& 4\end{array}$
\end{tabular}

\medskip

\begin{tabular}{ccc}
$\bL_8$&
\begin{minipage}{0.1\textwidth}
\begin{tikzpicture}
    [scale=0.6, e/.style={circle,draw,inner sep=0pt,minimum size=4pt}]
\node(5) at (0,1)[e]{};
\node(4) at (-.75,0.0)[e]{};
\node(3) at (-.25,0)[e]{};
\node(2) at (0.25,0)[e]{};
\node(1) at (.75,0)[e]{};
\node(0) at (0,-1)[e]{};
\node at (0,1.3){};
\draw(4)--(5);
\draw(3)--(5);
\draw(2)--(5);
\draw(1)--(5);
\draw(0)--(1);
\draw(0)--(2);
\draw(0)--(3);
\draw(0)--(4);
\end{tikzpicture}
\end{minipage}
&
$\setlength{\arraycolsep}{1pt}\begin{array}{c|cccccc}
    \bB_8& 0& 1& 2& 3& 4& 5\\\hline
   f(x)& 1& 2& 0& 4& 5& 3\\
   g(x)& 3& 5& 4& 0& 2& 1\end{array}$
\end{tabular}

\medskip

\begin{tabular}{ccc}
$\bL_9$&
\begin{minipage}{0.08\textwidth}
\begin{tikzpicture}
    [scale=0.6, e/.style={circle,draw,inner sep=0pt,minimum size=4pt}]
\node(6) at (0,1)[e]{};
\node(5) at (0.5,0.33)[e]{};
\node(4) at (-0.33,0)[e]{};
\node(3) at (-0.66,0)[e]{};
\node(2) at (0,0)[e]{};
\node(1) at (0.5,-0.33)[e]{};
\node(0) at (0,-1)[e]{};
\node at (0,1.3){};
\draw(5)--(6);
\draw(4)--(6);
\draw(3)--(6);
\draw(2)--(6);
\draw(1)--(5);
\draw(0)--(1);
\draw(0)--(2);
\draw(0)--(3);
\draw(0)--(4);
\end{tikzpicture}
\end{minipage}
&
$\setlength{\arraycolsep}{1pt}\begin{array}{c|cccccccccccccccc}
        \bB_9& 0& 1& 2& 3& 4& 5& 6& 7& 8& 9 & 10& 11& 12& 13& 14& 15\\\hline
    f(x)& 0& 0& 0& 0& 0& 0& 2& 1&  2&  1&   3&   4&   5&   3&   4&  5\\
   g(x)& 0& 0& 0& 0& 0& 0& 6&  7&  6&  7& 10& 11& 12& 10& 11& 12 \\
   h(x)&13&14&15&1&9&8&15&14&13& 15& 1&  9&    8&   8& 1&   9\end{array}$
%Method: overalgebras
\end{tabular}

\qquad\qquad \textcolor{MyDarkGreen}{Method: overalgebras}

\end{frame}

\begin{frame}

\begin{tabular}{ccc}
$\bL_{10}$&
\begin{minipage}{0.1\textwidth}
\begin{tikzpicture}
    [scale=.6, e/.style={circle,draw,inner sep=0pt,minimum size=4pt}]
\node(6) at (0.33,1)[e]{};
\node(5) at (0,0.33)[e]{};
\node(4) at (0.66,0.33)[e]{};
\node(3) at (-1,0)[e]{};
\node(2) at (-0.33,-0.33)[e]{};
\node(1) at (0.33,-0.33)[e]{};
\node(0) at (0,-1)[e]{};
\node at (0,1.3){};
\draw(5)--(6);
\draw(4)--(6);
\draw(3)--(6);
\draw(2)--(5);
\draw(1)--(4);
\draw(1)--(5);
\draw(0)--(1);
\draw(0)--(2);
\draw(0)--(3);
\end{tikzpicture}
\end{minipage}
&
\begin{tabular}{l}
\textbf{No finite algebra known with this}\\ 
\textbf{as its congruence lattice.}
\end{tabular}
\end{tabular}

\begin{tabular}{ccc}
$\bL_{11}$&
\begin{minipage}{0.08\textwidth}
\begin{tikzpicture}
    [xscale=.4, yscale=.6, e/.style={circle,draw,inner sep=0pt,minimum size=4pt}]
\node(6) at (0.33,1)[e]{};
\node(5) at (-0.33,0.33)[e]{};
\node(4) at (1,0.33)[e]{};
\node(3) at (0,0)[e]{};
\node(2) at (-1,-0.33)[e]{};
\node(1) at (0.33,-0.33)[e]{};
\node(0) at (-0.33,-1)[e]{};
\node at (0,1.3){};
\draw(5)--(6);
\draw(4)--(6);
\draw(3)--(5);
\draw(2)--(5);
\draw(1)--(3);
\draw(1)--(4);
\draw(0)--(1);
\draw(0)--(2);
\end{tikzpicture}
\end{minipage}
&
\begin{tabular}{l}
A finite algebra with 108 elements known.\\
\end{tabular}
\end{tabular}

\begin{tabular}{ccc}
$\bL_{12}$&
\begin{minipage}{0.07\textwidth}
\begin{tikzpicture}
    [scale=.6, e/.style={circle,draw,inner sep=0pt,minimum size=4pt}]
\node(6) at (0,1)[e]{};
\node(5) at (0.5,0.33)[e]{};
\node(4) at (-0.5,0.33)[e]{};
\node(3) at (0.0,0.0)[e]{};
\node(2) at (0.5,-0.33)[e]{};
\node(1) at (-0.5,-0.33)[e]{};
\node(0) at (0,-1)[e]{};
\node at (0,1.3){};
\draw(5)--(6);
\draw(4)--(6);
\draw(3)--(6);
\draw(2)--(5);
\draw(1)--(4);
\draw(0)--(1);
\draw(0)--(2);
\draw(0)--(3);
\end{tikzpicture}
\end{minipage}
&
$\setlength{\arraycolsep}{1pt}\begin{array}{c|ccccccccc}
   \bB_{12}& 0& 1& 2& 3& 4& 5& 6& 7& 8\\\hline
f(x)& 0& 0& 3& 3& 3& 6& 6& 6& 0\\
g(x)& 0& 0& 8& 8& 8& 1& 1& 1& 0\\
h(x)& 0& 5& 5& 4& 0& 0& 5& 4& 4\\
k(x)& 4& 2& 2& 3& 4& 4& 2& 3& 3\\
l(x)& 5& 5& 7& 7& 7& 6& 6& 6& 5\end{array}$
\end{tabular}

\end{frame}

\begin{frame}
\begin{tabular}{ccc}
$\bL_{13}$&
\begin{minipage}{0.07\textwidth}
\begin{tikzpicture}
    [scale=.6, e/.style={circle,draw,inner sep=0pt,minimum size=4pt}]
\node(6) at (0,1)[e]{};
\node(5) at (0.5,0.5)[e]{};
\node(4) at (0,0)[e]{};
\node(3) at (-0.66,0)[e]{};
\node(2) at (0.5,0)[e]{};
\node(1) at (0.5,-0.5)[e]{};
\node(0) at (0,-1)[e]{};
\node at (0,1.3){};
\draw(5)--(6);
\draw(4)--(6);
\draw(3)--(6);
\draw(2)--(5);
\draw(1)--(2);
\draw(0)--(1);
\draw(0)--(3);
\draw(0)--(4);
\end{tikzpicture}
\end{minipage}
&
{\footnotesize
$\setlength{\arraycolsep}{1pt}
\begin{array}{c|ccccccccccccccccccc}
    \bB_{13}& 0& 1& 2& 3& 4& 5& 6& 7& 8& 9&10&11&12&13&14&15&16&17&18\\\hline
f(x)& 0& 1& 2& 1& 2& 1& 0& 0& 1& 2&   2& 1&  0&  0&  1&  2&  1&  2& 0\\
g(x)& 0& 1& 2& 0& 0& 2& 2& 0& 3& 4&  0&  4& 4&   6&  5&  2&  6&  6& 2\\
h(x)& 0& 1& 2& 3& 4& 5& 6& 0& 1& 2&  4&  5& 6&   0& 1&   2&  3&  4& 6\\
k(x)& 7& 8& 9& 3&10&11&12&3&3& 3&  3&  3& 3& 11&11&11& 11&11&11\\
l(x)&13&14&15&16&17& 5&18&13&16&17&17&16&13&5&5&5& 5&  5& 5
 \end{array}$}
 %\begin{tabular}{l}
 %Method: \\overalgebras
 %\end{tabular}
\end{tabular}

\qquad\qquad \textcolor{MyDarkGreen}{Method: overalgebras}

\begin{tabular}{ccc}
$\bL_{14}$&
\begin{minipage}{0.08\textwidth}
\begin{tikzpicture}
    [scale=.6, e/.style={circle,draw,inner sep=0pt,minimum size=4pt}]
\node(6) at (0,1)[e]{};
\node(5) at (0.4,0.33)[e]{};
\node(4) at (0,0.33)[e]{};
\node(3) at (-0.8,0)[e]{};
\node(2) at (-0.4,0)[e]{};
\node(1) at (0.2,-0.33)[e]{};
\node(0) at (0,-1)[e]{};
\node at (0,1.3){};
\draw(5)--(6);
\draw(4)--(6);
\draw(3)--(6);
\draw(2)--(6);
\draw(1)--(4);
\draw(1)--(5);
\draw(0)--(1);
\draw(0)--(2);
\draw(0)--(3);
\end{tikzpicture}
\end{minipage}
&
\begin{tabular}{l}
Upper interval in Sub$(A_6)$,\\
algebra of size 90\end{tabular}
\end{tabular}

\begin{tabular}{ccc}
$\bL_{15}$&
\begin{minipage}{0.08\textwidth}
\begin{tikzpicture}
    [scale=.6, e/.style={circle,draw,inner sep=0pt,minimum size=4pt}]
\node(6) at (0,1)[e]{};
\node(5) at (0.2,0.33)[e]{};
\node(4) at (-0.4,0)[e]{};
\node(3) at (-0.8,0)[e]{};
\node(2) at (0,-0.33)[e]{};
\node(1) at (0.4,-0.33)[e]{};
\node(0) at (0,-1)[e]{};
\node at (0,1.3){};
\draw(5)--(6);
\draw(4)--(6);
\draw(3)--(6);
\draw(2)--(5);
\draw(1)--(5);
\draw(0)--(4);
\draw(0)--(3);
\draw(0)--(2);
\draw(0)--(1);
\end{tikzpicture}
\end{minipage}
&
$\setlength{\arraycolsep}{1pt}\begin{array}{c|cccc}
    \bB_{15}& 0& 1& 2& 3\\\hline
   f(x)& 1& 0& 3& 2\end{array}$
\end{tabular}

\begin{tabular}{ccc}
$\bL_{16}$&
\begin{minipage}{0.07\textwidth}
\begin{tikzpicture}
    [scale=.6, e/.style={circle,draw,inner sep=0pt,minimum size=4pt}]
\node(6) at (0,1)[e]{};
\node(5) at (-0.1,0.33)[e]{};
\node(4) at (0.5,0.33)[e]{};
\node(3) at (0.2,0.33)[e]{};
\node(2) at (-0.66,0)[e]{};
\node(1) at (0.25,-0.33)[e]{};
\node(0) at (0,-1)[e]{};
\node at (0,1.3){};
\draw(5)--(6);
\draw(4)--(6);
\draw(3)--(6);
\draw(2)--(6);
\draw(1)--(3);
\draw(1)--(4);
\draw(1)--(5);
\draw(0)--(1);
\draw(0)--(2);
\end{tikzpicture}
\end{minipage}
&
\begin{tabular}{l}
Upper interval in Sub$(C_2.A_6)$\\
algebra of size 180\end{tabular}
\end{tabular}

\end{frame}

\begin{frame}

\begin{tabular}{ccc}
$\bL_{17}$&
\begin{minipage}{0.07\textwidth}
\begin{tikzpicture}
    [scale=.6, e/.style={circle,draw,inner sep=0pt,minimum size=4pt}]
\node(6) at (0,1.0)[e]{};
\node(5) at (0.25,0.33)[e]{};
\node(4) at (-0.66,0)[e]{};
\node(3) at (0.2,-0.33)[e]{};
\node(2) at (-0.1,-0.33)[e]{};
\node(1) at (0.5,-0.33)[e]{};
\node(0) at (0,-1)[e]{};
\node at (0,1.3){};
\draw(5)--(6);
\draw(4)--(6);
\draw(3)--(5);
\draw(2)--(5);
\draw(1)--(5);
\draw(0)--(4);
\draw(0)--(3);
\draw(0)--(2);
\draw(0)--(1);
\end{tikzpicture}
\end{minipage}
&
$\setlength{\arraycolsep}{1pt}\begin{array}{c|cccccccccccc}
       \bB_{17}& 0& 1& 2& 3& 4& 5& 6& 7& 8& 9& 10& 11\\\hline
    f(x)& 1& 0& 3& 2& 5& 4& 7& 6& 9& 8& 11& 10\\
   g(x)& 4& 7& 5& 6& 8&11& 9&10& 0& 3&  1&  2\\
   h(x)& 0& 0& 0& 0& 5& 5& 5& 5&10&10&10&10\end{array}$

\end{tabular}

\qquad\qquad\qquad\textcolor{MyDarkGreen}{Method: filter-ideal in Sub$(A_4)$}


\begin{tabular}{ccc}
$\bL_{18}$&
\begin{minipage}{0.07\textwidth}
\begin{tikzpicture}
    [scale=.6, e/.style={circle,draw,inner sep=0pt,minimum size=4pt}]
\node(6) at (0,1)[e]{};
\node(5) at (0.5,0.33)[e]{};
\node(4) at (-0.5,0.33)[e]{};
\node(3) at (0,0.33)[e]{};
\node(2) at (0.5,-0.33)[e]{};
\node(1) at (-0.5,-0.33)[e]{};
\node(0) at (0,-1)[e]{};
\node at (0,1.3){};
\draw(5)--(6);
\draw(4)--(6);
\draw(3)--(6);
\draw(2)--(5);
\draw(1)--(3);
\draw(1)--(4);
\draw(0)--(1);
\draw(0)--(2);
\end{tikzpicture}
\end{minipage}
&
\begin{tabular}{l}
Dual of 19, no explicit\\
small representation known\end{tabular}
\end{tabular}

\begin{tabular}{ccc}
$\bL_{19}$&
\begin{minipage}{0.07\textwidth}
\begin{tikzpicture}
    [scale=.6, e/.style={circle,draw,inner sep=0pt,minimum size=4pt}]
\node(6) at (0,1)[e]{};
\node(5) at (-0.5,0.33)[e]{};
\node(4) at (0.5,0.33)[e]{};
\node(3) at (-0.5,-0.33)[e]{};
\node(2) at (0,-0.33)[e]{};
\node(1) at (0.5,-0.33)[e]{};
\node(0) at (0,-1)[e]{};
\node at (0,1.3){};
\draw(5)--(6);
\draw(4)--(6);
\draw(3)--(5);
\draw(2)--(5);
\draw(1)--(4);
\draw(0)--(3);
\draw(0)--(2);
\draw(0)--(1);
\end{tikzpicture}
\end{minipage}
&
$\setlength{\arraycolsep}{1pt}\begin{array}{c|cccccccc}
   \bB_{19}& 0& 1& 2& 3& 4& 5& 6& 7\\\hline
f(x)& 0& 1& 1& 0& 4& 5& 5& 4\\
g(x)& 0& 2& 3& 1& 0& 2& 3& 1\\
h(x)& 7& 6& 6& 7& 3& 2& 2& 3\end{array}$
\end{tabular}

\begin{tabular}{ccc}
$\bL_{20}$&
\begin{minipage}{0.07\textwidth}
\begin{tikzpicture}
    [scale=.6, e/.style={circle,draw,inner sep=0pt,minimum size=4pt}]
\node(6) at (0,1)[e]{};
\node(5) at (0.44,0.66)[e]{};
\node(4) at (0,0.33)[e]{};
\node(3) at (-0.66,0)[e]{};
\node(2) at (0.44,0)[e]{};
\node(1) at (0,-0.33)[e]{};
\node(0) at (0,-1)[e]{};
\node at (0,1.3){};
\draw(5)--(6);
\draw(4)--(6);
\draw(3)--(6);
\draw(2)--(5);
\draw(1)--(2);
\draw(1)--(4);
\draw(0)--(1);
\draw(0)--(3);
\end{tikzpicture}
\end{minipage}
&
\begin{tabular}{l}
\textcolor{MyDarkGreen}{Method: filter-ideal in SmallGroup(216,153) in GAP}\\
\end{tabular}
\end{tabular}

\end{frame}

\begin{frame}

\begin{tabular}{ccc}
$\bL_{21}$&
\begin{minipage}{0.07\textwidth}
\begin{tikzpicture}
    [scale=.6, e/.style={circle,draw,inner sep=0pt,minimum size=4pt}]
\node(6) at (0,1)[e]{};
\node(5) at (0,0.33)[e]{};
\node(4) at (0.44,0)[e]{};
\node(3) at (-0.66,0)[e]{};
\node(2) at (0,-0.33)[e]{};
\node(1) at (0.44,-0.66)[e]{};
\node(0) at (0,-1)[e]{};
\node at (0,1.3){};
\draw(5)--(6);
\draw(4)--(5);
\draw(3)--(6);
\draw(2)--(5);
\draw(1)--(4);
\draw(0)--(3);
\draw(0)--(2);
\draw(0)--(1);
\end{tikzpicture}
\end{minipage}
&
$\setlength{\arraycolsep}{1pt}\begin{array}{c|ccccccccc}
   \bB_{21}& 0& 1& 2& 3& 4& 5& 6& 7& 8\\\hline
f(x)& 3& 3& 4& 8& 8& 2& 2& 3& 4\\
g(x)& 0& 0& 6& 1& 1& 0& 0& 5& 6\\
h(x)& 4& 5& 5& 7& 8& 8& 7& 4& 4\end{array}$
\end{tabular}

\begin{tabular}{ccc}
$\bL_{22}$&
\begin{minipage}{0.07\textwidth}
\begin{tikzpicture}
    [scale=.6, e/.style={circle,draw,inner sep=0pt,minimum size=4pt}]
\node(6) at (0,1)[e]{};
\node(5) at (0.5,0.5)[e]{};
\node(4) at (0,0.5)[e]{};
\node(3) at (-0.66,0)[e]{};
\node(2) at (0.5,0)[e]{};
\node(1) at (0.5,-0.5)[e]{};
\node(0) at (0,-1)[e]{};
\node at (0,1.3){};
\draw(5)--(6);
\draw(4)--(6);
\draw(3)--(6);
\draw(2)--(4);
\draw(2)--(5);
\draw(1)--(2);
\draw(0)--(1);
\draw(0)--(3);
\end{tikzpicture}
\end{minipage}
&
\begin{tabular}{l}
\textcolor{MyDarkGreen}{Dual of 23, no explicit small}\\
\textcolor{MyDarkGreen}{representation known}
\end{tabular}
\end{tabular}

\begin{tabular}{ccc}
$\bL_{23}$&
\begin{minipage}{0.07\textwidth}
\begin{tikzpicture}
    [scale=.6, e/.style={circle,draw,inner sep=0pt,minimum size=4pt}]
\node(6) at (0,1)[e]{};
\node(5) at (0.5,0.5)[e]{};
\node(4) at (0.5,0)[e]{};
\node(3) at (-0.66,0)[e]{};
\node(2) at (0.5,-0.5)[e]{};
\node(1) at (0,-0.5)[e]{};
\node(0) at (0,-1)[e]{};
\node at (0,1.3){};
\draw(5)--(6);
\draw(4)--(5);
\draw(3)--(6);
\draw(2)--(4);
\draw(1)--(4);
\draw(0)--(3);
\draw(0)--(2);
\draw(0)--(1);
\end{tikzpicture}
\end{minipage}
&
$\setlength{\arraycolsep}{1pt}\begin{array}{c|cccccc}
      \bB_{23}& 0& 1& 2& 3& 4& 5\\\hline
   f(x)& 0& 1& 0& 1& 4& 4\\
   g(x)& 1& 1& 3& 3& 4& 5\\
   h(x)& 3& 2& 3& 2& 5& 5\\
   k(x)& 4& 1& 5& 3& 4& 5\end{array}$
\end{tabular}

\medskip

\begin{tabular}{ccc}
$\bL_{24}$&
\begin{minipage}{0.07\textwidth}
\begin{tikzpicture}
    [scale=.6, e/.style={circle,draw,inner sep=0pt,minimum size=4pt}]
\node(6) at (0,1)[e]{};
\node(5) at (0,0.33)[e]{};
\node(4) at (0.5,0.33)[e]{};
\node(3) at (-0.5,0.33)[e]{};
\node(2) at (0.5,-0.33)[e]{};
\node(1) at (-0.5,-0.33)[e]{};
\node(0) at (0,-1)[e]{};
\node at (0,1.3){};
\draw(5)--(6);
\draw(4)--(6);
\draw(3)--(6);
\draw(2)--(4);
\draw(2)--(5);
\draw(1)--(3);
\draw(1)--(5);
\draw(0)--(1);
\draw(0)--(2);
\end{tikzpicture}
\end{minipage}
&
$\setlength{\arraycolsep}{1pt}\begin{array}{c|cccc}
      \bB_{24}& 0& 1& 2& 3\\\hline
   f(x)& 1& 1& 2& 2\\
   g(x)& 2& 3& 3& 2\end{array}$
\end{tabular}

\end{frame}

\begin{frame}


\begin{tabular}{ccc}
$\bL_{25}$&
\begin{minipage}{0.07\textwidth}
\begin{tikzpicture}
    [scale=.6, e/.style={circle,draw,inner sep=0pt,minimum size=4pt}]
\node(6) at (0,1)[e]{};
\node(5) at (-0.5,0.33)[e]{};
\node(4) at (0.5,0.33)[e]{};
\node(3) at (-0.5,-0.33)[e]{};
\node(2) at (0.5,-0.33)[e]{};
\node(1) at (0,-0.33)[e]{};
\node(0) at (0,-1)[e]{};
\node at (0,1.3){};
\draw(5)--(6);
\draw(4)--(6);
\draw(3)--(5);
\draw(2)--(4);
\draw(1)--(5);
\draw(1)--(4);
\draw(0)--(3);
\draw(0)--(2);
\draw(0)--(1);
\end{tikzpicture}
\end{minipage}
&
$\setlength{\arraycolsep}{1pt}\begin{array}{c|ccccc}
      \bB_{25}& 0& 1& 2& 3& 4\\\hline
   f(x)&0& 0& 2& 2& 2\\
   g(x)&0& 1& 0& 1& 1\\
   h(x)&1& 1& 4& 4& 4\\
   k(x)&2& 3& 2& 3& 3\end{array}$
\end{tabular}


\medskip

\begin{tabular}{ccc}
$\bL_{26}$&
\begin{minipage}{0.07\textwidth}
\begin{tikzpicture}
    [scale=.6, e/.style={circle,draw,inner sep=0pt,minimum size=4pt}]
\node(6) at (0,1)[e]{};
\node(5) at (0.25,0.5)[e]{};
\node(4) at (-0.66,0)[e]{};
\node(3) at (0,0)[e]{};
\node(2) at (0.5,0)[e]{};
\node(1) at (0.25,-0.5)[e]{};
\node(0) at (0,-1)[e]{};
\node at (0,1.3){};
\draw(5)--(6);
\draw(4)--(6);
\draw(3)--(5);
\draw(2)--(5);
\draw(1)--(2);
\draw(1)--(3);
\draw(0)--(1);
\draw(0)--(4);
\end{tikzpicture}
\end{minipage}
&
$\setlength{\arraycolsep}{1pt}\begin{array}{c|cccccc}
      \bB_{26}& 0& 1& 2& 3& 4& 5\\\hline
   f(x)&1& 0& 3& 2& 0& 2\\
   g(x)&4& 4& 5& 5& 1& 3\\
   h(x)&0& 0& 0& 0& 1& 1\\
   k(x)&3& 5& 3& 5& 3& 3\end{array}$
\end{tabular}

\end{frame}

\begin{frame}


\begin{tabular}{ccc}
$\bL_{27}$&
\begin{minipage}{0.07\textwidth}
\begin{tikzpicture}
    [scale=.6, e/.style={circle,draw,inner sep=0pt,minimum size=4pt}]
\node(6) at (0,1)[e]{};
\node(5) at (0.5,0.5)[e]{};
\node(4) at (-0.5,0.33)[e]{};
\node(3) at (0.5,0.0)[e]{};
\node(2) at (-0.5,-0.33)[e]{};
\node(1) at (0.5,-0.5)[e]{};
\node(0) at (0,-1)[e]{};
\node at (0,1.3){};
\draw(5)--(6);
\draw(4)--(6);
\draw(3)--(5);
\draw(2)--(4);
\draw(1)--(3);
\draw(0)--(1);
\draw(0)--(2);
\end{tikzpicture}
\end{minipage}
&
$\setlength{\arraycolsep}{1pt}\begin{array}{c|cccccccccccccccc}
        \bB_{27}& 0& 1& 2& 3& 4& 5& 6& 7& 8& 9 & 10& 11& 12& 13& 14& 15\\\hline
   f(x)&0& 1& 2& 3& 4& 5& 0& 0& 0& 0& 0& 2& 2& 2& 2& 2\\
   g(x)&4& 5& 3& 4& 5& 3& 5& 3& 4& 5& 3& 4& 5& 4& 5& 3\\
   h(x)&2& 2& 1& 5& 5& 4& 2& 1& 5& 5& 4& 2& 2& 5& 5& 4\\
   k(x)&3& 4& 4& 0& 1& 1& 4& 4& 0& 1& 1& 3& 4& 0& 1& 1\\
   l(x)&0& 6& 7& 8& 9& 10& 6& 7& 8& 9& 10& 0& 6& 8& 9& 10\\
   m(x)&11& 12& 2& 13& 14& 15& 12& 2& 13& 14& 15& 11& 12& 13& 14& 15
\end{array}$
 %\begin{tabular}{l}
 %\textcolor{MyDarkGreen}{Method:} \\overalgebras
 %\end{tabular}
\end{tabular}

\qquad\qquad\qquad\textcolor{MyDarkGreen}{Method: overalgebras}


\medskip

\begin{tabular}{ccc}
$\bL_{28}$&
\begin{minipage}{0.07\textwidth}
\begin{tikzpicture}
    [scale=.6, e/.style={circle,draw,inner sep=0pt,minimum size=4pt}]
\node(6) at (0,1)[e]{};
\node(5) at (0.5,0.6)[e]{};
\node(4) at (-0.5,0)[e]{};
\node(3) at (0.5,0.2)[e]{};
\node(2) at (0.5,-0.2)[e]{};
\node(1) at (0.5,-0.6)[e]{};
\node(0) at (0,-1)[e]{};
\node at (0,1.3){};
\draw(5)--(6);
\draw(4)--(6);
\draw(3)--(5);
\draw(2)--(3);
\draw(1)--(2);
\draw(0)--(1);
\draw(0)--(4);
\end{tikzpicture}
\end{minipage}
&
$\setlength{\arraycolsep}{1pt}\begin{array}{c|cccccccccccccccc}
        \bB_{28}& 0& 1& 2& 3& 4& 5& 6& 7& 8& 9 & 10& 11& 12& 13& 14& 15\\\hline
   f(x)&0& 1& 2& 3& 4& 5& 0& 0& 0& 0& 0& 2& 2& 2& 2& 2\\
   g(x)&3& 3& 4& 3& 3& 4& 3& 4& 3& 3& 4& 3& 3& 3& 3& 4\\
   h(x)&1& 0& 0& 4& 3& 3& 0& 0& 4& 3& 3& 1& 0& 4& 3& 3\\
   k(x)&4& 5& 5& 1& 2& 2& 5& 5& 1& 2& 2& 4& 5& 1& 2& 2\\
   l(x)&0& 6& 7& 8& 9& 10& 6& 7& 8& 9& 10& 0& 6& 8& 9& 10\\
   m(x)&11& 12& 2& 13& 14& 15& 12& 2& 13& 14& 15& 11& 12& 13& 14& 15
\end{array}$
% \begin{tabular}{l}
 %\textcolor{MyDarkGreen}{Method: \\overalgebras}
% \end{tabular}
\end{tabular}

\qquad\qquad\qquad\textcolor{MyDarkGreen}{Method: overalgebras}


\end{frame}

\begin{frame}



\begin{tabular}{ccc}
$\bL_{29}$&
\begin{minipage}{0.07\textwidth}
\begin{tikzpicture}
    [scale=.6, e/.style={circle,draw,inner sep=0pt,minimum size=4pt}]
\node(6) at (0.,1)[e]{};
\node(5) at (0.5,0.5)[e]{};
\node(4) at (-0.66,0.25)[e]{};
\node(3) at (0.5,0)[e]{};
\node(2) at (-0.66,-0.25)[e]{};
\node(1) at (0,-0.5)[e]{};
\node(0) at (0,-1)[e]{};
\node at (0,1.3){};
\draw(5)--(6);
\draw(4)--(6);
\draw(3)--(5);
\draw(2)--(4);
\draw(1)--(3);
\draw(1)--(4);
\draw(0)--(1);
\draw(0)--(2);
\end{tikzpicture}
\end{minipage}
&
$\setlength{\arraycolsep}{1pt}\begin{array}{c|ccccc}
      \bB_{29}& 0& 1& 2& 3& 4\\\hline
   f(x)&1& 0& 3& 2& 2\\
   g(x)&2& 4& 2& 4& 3\end{array}$
\end{tabular}

\medskip

\begin{tabular}{ccc}
$\bL_{30}$&
\begin{minipage}{0.07\textwidth}
\begin{tikzpicture}
    [scale=.6, e/.style={circle,draw,inner sep=0pt,minimum size=4pt}]
\node(6) at (0,1)[e]{};
\node(5) at (0,0.5)[e]{};
\node(4) at (-0.66,0.25)[e]{};
\node(3) at (0.5,0)[e]{};
\node(2) at (-0.66,-0.25)[e]{};
\node(1) at (0.5,-0.5)[e]{};
\node(0) at (0,-1)[e]{};
\node at (0,1.3){};
\draw(5)--(6);
\draw(4)--(6);
\draw(3)--(5);
\draw(2)--(5);
\draw(2)--(4);
\draw(1)--(3);
\draw(0)--(2);
\draw(0)--(1);
\end{tikzpicture}
\end{minipage}
&
$\setlength{\arraycolsep}{1pt}\begin{array}{c|ccccc}
      \bB_{30}& 0& 1& 2& 3& 4\\\hline
   f(x)&0& 3& 4& 3& 4\\
   g(x)&2& 2& 1& 4& 3\end{array}$
\end{tabular}

\medskip

\begin{tabular}{ccc}
$\bL_{31}$&
\begin{minipage}{0.07\textwidth}
\begin{tikzpicture}
    [scale=.6, e/.style={circle,draw,inner sep=0pt,minimum size=4pt}]
\node(6) at (0,1)[e]{};
\node(5) at (0.5,0.5)[e]{};
\node(4) at (-0.66,0.25)[e]{};
\node(3) at (0.5,0)[e]{};
\node(2) at (0.5,-0.5)[e]{};
\node(1) at (0,-0.5)[e]{};
\node(0) at (0,-1)[e]{};
\node at (0,1.3){};
\draw(5)--(6);
\draw(4)--(6);
\draw(3)--(5);
\draw(2)--(3);
\draw(1)--(3);
\draw(1)--(4);
\draw(0)--(1);
\draw(0)--(2);
\end{tikzpicture}
\end{minipage}
&
$\setlength{\arraycolsep}{1pt}\begin{array}{c|ccccc}
   \bB_{31}& 0& 1& 2& 3& 4\\\hline
   f(x)&0& 1& 1& 0& 0\\
   g(x)&1& 1& 2& 2& 2\\
   h(x)&3& 2& 2& 4& 4\end{array}$
\end{tabular}

\medskip

\begin{tabular}{ccc}
$\bL_{32}$&
\begin{minipage}{0.07\textwidth}
\begin{tikzpicture}
    [scale=.6, e/.style={circle,draw,inner sep=0pt,minimum size=4pt}]
\node(6) at (0,1)[e]{};
\node(5) at (0,0.5)[e]{};
\node(4) at (0.5,0.5)[e]{};
\node(3) at (0.5,0)[e]{};
\node(2) at (-0.66,-0.25)[e]{};
\node(1) at (0.5,-0.5)[e]{};
\node(0) at (0,-1)[e]{};
\node at (0,1.3){};
\draw(5)--(6);
\draw(4)--(6);
\draw(3)--(5);
\draw(3)--(4);
\draw(2)--(5);
\draw(1)--(3);
\draw(0)--(2);
\draw(0)--(1);
\end{tikzpicture}
\end{minipage}
&
$\setlength{\arraycolsep}{1pt}\begin{array}{c|ccccc}
      \bB_{32}& 0& 1& 2& 3& 4\\\hline
   f(x)&0& 1& 1& 3& 3\\
   g(x)&1& 2& 2& 4& 4\\
   h(x)&3& 3& 4& 3& 4\end{array}$
\end{tabular}

\end{frame}

\begin{frame}


\begin{tabular}{ccc}
$\bL_{33}$&
\begin{minipage}{0.12\textwidth}
\begin{tikzpicture}
    [scale=.6, e/.style={circle,draw,inner sep=0pt,minimum size=4pt}]
\node(6) at (0,1)[e]{};
\node(5) at (1,0)[e]{};
\node(4) at (0.5,0)[e]{};
\node(3) at (0,0)[e]{};
\node(2) at (-0.5,0)[e]{};
\node(1) at (-1,0)[e]{};
\node(0) at (0,-1)[e]{};
\node at (0,1.3){};
\draw(5)--(6);
\draw(4)--(6);
\draw(3)--(6);
\draw(2)--(6);
\draw(1)--(6);
\draw(0)--(1);
\draw(0)--(2);
\draw(0)--(3);
\draw(0)--(4);
\draw(0)--(5);
\end{tikzpicture}
\end{minipage}
&
{\footnotesize
$\setlength{\arraycolsep}{1pt}\begin{array}{c|cccccccccccccccc}
        \bB_{33}& 0& 1& 2& 3& 4& 5& 6& 7& 8& 9 & 10& 11& 12& 13& 14& 15\\\hline
   f(x)&1& 3& 2& 0& 9& 11& 10& 8& 13& 15& 14& 12& 5& 7& 6& 4\\
   g(x)&11& 8& 10& 9& 7& 4& 6& 5& 15& 12& 14& 13& 3& 0& 2& 1\\
   h(x)&14& 15& 12& 13& 10& 11& 8& 9& 6& 7& 4& 5& 2& 3& 0& 1\end{array}$}
\end{tabular}

\begin{tabular}{ccc}
$\bL_{34}$&
\begin{minipage}{0.07\textwidth}
\begin{tikzpicture}
    [scale=.6, e/.style={circle,draw,inner sep=0pt,minimum size=4pt}]
\node(6) at (0,1)[e]{};
\node(5) at (-0.33,0.33)[e]{};
\node(4) at (0.33,0.33)[e]{};
\node(3) at (0,0.33)[e]{};
\node(2) at (-0.66,-0.33)[e]{};
\node(1) at (0,-0.33)[e]{};
\node(0) at (-0.33,-1)[e]{};
\node at (0,1.3){};
\draw(5)--(6);
\draw(4)--(6);
\draw(3)--(6);
\draw(2)--(5);
\draw(1)--(3);
\draw(1)--(4);
\draw(1)--(5);
\draw(0)--(1);
\draw(0)--(2);
\end{tikzpicture}
\end{minipage}
&
$\setlength{\arraycolsep}{1pt}\begin{array}{c|cccc}
      \bB_{34}& 0& 1& 2& 3\\\hline
   f(x)&0& 1& 3& 2\end{array}$
\end{tabular}

\begin{tabular}{ccc}
$\bL_{35}$&
\begin{minipage}{0.07\textwidth}
\begin{tikzpicture}
    [scale=.6, e/.style={circle,draw,inner sep=0pt,minimum size=4pt}]
\node(6) at (-0.33,1)[e]{};
\node(5) at (0,0.33)[e]{};
\node(4) at (-0.66,0.33)[e]{};
\node(3) at (0.33,-0.33)[e]{};
\node(2) at (0,-0.33)[e]{};
\node(1) at (-0.33,-0.33)[e]{};
\node(0) at (0,-1)[e]{};
\node at (0,1.3){};
\draw(5)--(6);
\draw(4)--(6);
\draw(3)--(5);
\draw(2)--(5);
\draw(1)--(5);
\draw(1)--(4);
\draw(0)--(3);
\draw(0)--(2);
\draw(0)--(1);
\end{tikzpicture}
\end{minipage}
&
$\setlength{\arraycolsep}{1pt}\begin{array}{c|cccc}
      \bB_{35}& 0& 1& 2& 3\\\hline
   f(x)&1& 1& 2& 3\\
   g(x)&2& 3& 3& 3\end{array}$
\end{tabular}


\end{frame}

\begin{frame}
\frametitle{Problem}

\begin{itemize}
\item
What about nonunary algebras?
\end{itemize}

\uncover<2->{
\begin{problem}
Which (finite) lattices can be represented as $\Con  A$, 
where $\alg A$ has a Taylor term?
\end{problem}
}

\end{frame}



\begin{frame}
\frametitle{Resources}


\begin{itemize}
\item
\textcolor{MyDarkGreen}{SmallLatticeReps.ua}, a UACalc file with 
most of the $\alg B_i$'s:
\[
\texttt{http://math.hawaii.edu/$\sim$ralph/}
\]
\item
The slides of this talk are there too.
\item
SmallLatticeReps.ua and other algebra files:
\[
\texttt{https://github.com/UACalc/SmallAlgebras/}
\]
\item
Get UACalc at
\[
\texttt{http://uacalc.org/}
\]
\item
Source at 
\[
\texttt{https://github.com/UACalc/uacalcsrc/}
\]
\end{itemize}



\end{frame}


\end{document}
 
  
%%%%%%%%%%%%%%%%%%%%%%%%%%%%%%%
\begin{figure}[htb]
%\renewcommand{\dotsize}{1.5pt}
\begin{center}
  \begin{tikzpicture}[scale=1]
    \input{inputs/tikz/L14.tex}
    %    \draw (bottom) node [right]{$L_{11}$};
  \end{tikzpicture}
  \hskip15mm
  \begin{tikzpicture}[scale=1]
    \input{inputs/tikz/L15.tex}
    %    \draw (bottom) node [right]{$L_{11}$};
  \end{tikzpicture}
\end{center}
\caption{$\bL_{14}$ (on the left) and $\bL_{15}$}\label{fig:L14and15}
\end{figure}



\end{frame}



\end{document}




