\frame[label=methods]{
  \frametitle{How to prove a finite lattice is representable}
  %% \begin{columns}
  %%   \begin{column}{0.8\textwidth}
\uncover<2->{  \begin{block}{1. use closure properties}{Relate the given lattice to other lattices known to be representable.}

\vskip2mm

    \begin{itemize} %[label=$\diamond$]
    \item If $L$ is representable, so is 
      \begin{enumerate}[a.] %[label=$\circ$]
    %%   \item[$\cdot$] the dual of $L$ (Kurzweil, Netter)\\[4pt]
    %%   \item[$\cdot$] any interval sublattice of $L$\\[8pt]
    %%   \item[$\cdot$] any sublattice formed from the union of a principal filter and principal idea of $L$ (Snow) \\[8pt] 
    %%   \end{itemize}
    %% \item If $L_1$ and $L_2$ are representable, so is\\[4pt]
    %%   \begin{itemize} %[label=$\circ$]
    %%   \item[$\cdot$]  the direct product of $L_1$ and $L_2$ (\Tuma) \\[4pt]
    %%   \item[$\cdot$]  the ordinal sum of $L_1$ and $L_2$ (McKenzie, Snow)\\[4pt]
    %%   \item[$\cdot$]  the parallel sum of $L_1$ and $L_2$ (Snow)
      \item the dual of $L$ {\footnotesize (Kurzweil 1985, Netter)}\\[4pt]
      \item any interval sublattice of $L$ {\footnotesize (follows from A.)}\\[4pt]
      \item any sublattice that is the union of a principal filter and principal idea of $L$ {\footnotesize (Snow, 2000)} \\[8pt] 
      \end{enumerate}
\uncover<2->{
    \item If $L_1$ and $L_2$ are representable, so is\\[4pt]
      \begin{enumerate}[1.]
      \item  the direct product of $L_1$ and $L_2$ {\footnotesize (\Tuma 1989)} \\[4pt]
      \item  the ordinal sum of $L_1$ and $L_2$ {\footnotesize (McKenzie 1984, Snow 2000)}\\[4pt]
      \item  the parallel sum of $L_1$ and $L_2$ {\footnotesize (Snow 2000)}
      \end{enumerate}
}
    \end{itemize}
  \end{block}}
%\visible<2->{Subdirect products?}\\[4pt]
%\visible<3->{Homomorphic images?}

}

\frame[label=methods]{
  \frametitle{How to prove a finite lattice is representable}
  %% \begin{columns}
  %%   \begin{column}{0.8\textwidth}
  \begin{block}{2. the closure method}{Find a ``closed'' representation of $L$ in $\Eq(X)$.}
      
    \begin{itemize}
    \item For $L\leq \Eq(X)$ define
      \[
      \lambda(L) = \{f\in X^X: (\forall \theta \in L) \; f(\theta) \subseteq \theta \}
      \]
    \item For $F\subseteq X^X$ define
      \[
      \rho(F) = \{\theta \in \Eq(X) :   (\forall f\in F) \; f(\theta) \subseteq \theta\}
      \]
    %% \item[] For every $L \leq \Eq(X)$ we have $L \subseteq \rho \lambda (L)$.
      %% \vskip2mm
    \item The map $\rho \lambda$ is a \emph{closure operator} on $\Sub[\Eq(X)]$.\\
      {\small (idempotent, extensive, order preserving)}
      \vskip2mm
    \end{itemize}
  \end{block}
\uncover<2->{\begin{theorem}
A lattice $L\leq \Eq(X)$ is a congruence lattice if and only if it is \emph{closed},\\ 
i.e. $\rho\lambda(L) = L$, in which case $L = \Con \<X, \lambda(L)\>$.
  \end{theorem}}
\uncover<3->{\alert{Example:} $M_3 \cong L \leq \Eq(5)$}
}


\frame[label=methods]{
  \frametitle{How to prove a finite lattice is representable}
  \begin{block}{3. the G-set method}{Find $L$ as an interval in a subgroup lattice of a finite group.}

    \vskip3mm

    If $H\leq G$ are finite groups, then the filter above $H$ in $\Sub(G)$,
    \[
      \lb H,G \rb := \{K : H\leq K \leq G\},
      \]
      is isomorphic to $\Con\<G/H, \bar{G}\>$.

  \end{block}

  \vskip4mm

  %% \begin{block}{4. the filter+ideal method}{Find $L$ as the union of a filter and ideal in a representable lattice.}
  %% \end{block}
\uncover<2->{
  \begin{block}{4. the rabbit ears method {\footnotesize (aka overalgebras, aka expansion-extension)}}{Build the required algebra by gluing together isomorphic copies of an algebra and adding new operations.}
  \end{block}}

}



