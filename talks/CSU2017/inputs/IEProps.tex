\begin{frame}[fragile,label=Conclusion,shrink=5]{The exceptional seven element lattice}
  \frametitle{Interval enforceable properties}
  \begin{columns}
    \begin{column}{0.7\textwidth}
      \begin{itemize}
      %% \item  $L_7$ cannot be obtained using the overalgebra construction.
      %%   \vskip2mm
      \item<1-> A minimal representation of $L_7$ must come from a transitive $G$-set.
        \vskip2mm
      \item<2-> Suppose $L_7 \cong \lb H, G \rb$ for some finite groups $H<G$.\\[4pt]
        What can we say about the group $G$?
        \vskip2mm
      \item<3-> If we prove $G$ must have certain properties, then \FLRP\ has a
        positive answer iff every finite lattice is an interval
        in the subgroup lattice of a group \emph{satisfying all of these properties}.
        \note{
          \begin{itemize}
          \item Define $\core_G(H)$.  It is the largest normal subgroup in $G$
          contained in $H$.  It is also the kernel of the action of $G$ on the
          cosets of $H$.  
        \item $L_7 \cong \lb H, G \rb$, assume WLOG $H$ is \alert{core-free} in $G$.\\  
          Otherwise $L_7\cong \lb G/\core_G(H), H/\core_G(H) \rb$, and $H/\core_G(H)$ c.f. in $G/\core_G(H)$.
          \end{itemize}
        }
      \end{itemize}
    \end{column}
    \begin{column}{0.2\textwidth}
      \begin{center}
        \begin{tikzpicture}[scale=.5]
          %%  The elusive winged-2x3 %%
      \node (01) at (0,1)  [draw, circle, inner sep=\dotsize] {};
      \foreach \j in {0,2} 
      { \node (1\j) at (1,\j)  [draw, circle, inner sep=\dotsize] {};}

      \foreach \j in {1,3} 
      { \node (2\j) at (2,\j)  [draw, circle, inner sep=\dotsize] {};}
      { \node (32) at (3,2)  [draw, circle, inner sep=\dotsize] {};}
      \draw[semithick] (10) to (01) to (12) to (23) to (32) to (21) to (10) (21) to (12);
      { \node (m11) at (-1,1)  [draw, circle, inner sep=\dotsize] {};}
      \draw[semithick] (10) to (m11) to (23);


          \alt<1>{\draw (1,-.8) node {$L_7$};}{\draw (1,-.6) node {$H$};}
          \uncover<2->{\draw (2,3.6) node {$G$};}
        \end{tikzpicture}
      \end{center}
    \end{column}
  \end{columns}

  \only<4->{
\begin{block}{Proposition}
      \label{thm:except-seven-elem}
      Suppose $H<G$, \hskip2mm $\core_G(H) = 1$, \hskip2mm $L_7 \cong \lb H,G \rb$.
      \begin{enumerate}[(i)]
      \item<4-> $G$ is a primitive permutation group.
      \item<4-> If $N\ssubnormal G$, then $C_G(N) = 1$.
      \item<4-> $G$ contains no non-trivial abelian normal subgroup.
      \item<4-> $G$ is not solvable.
      \item<4-> $G$ is subdirectly irreducible.
      \item<4-> With the possible exception of at most one maximal subgroup, %, $M_1$ or $M_2$,
        all proper subgroups in the interval $\lb H,G \rb$ are core-free. 
      \end{enumerate}
    \end{block}
  }
  \note{  It is obvious that 
    \begin{itemize}
    \item (ii) $\Rightarrow$ (iii) $\Rightarrow$ (iv), and  
    \item (ii) $\Rightarrow$ (v), but we include these for
      emphasis;
    \item the hard work is in proving (ii) and (vi), but
      the main goal is the pair of restrictions (iii) and (v), which allow us to rule
      out a number of the O'Nan-Scott types describing primitive permutation
      groups. 
  \end{itemize}}
\end{frame}
