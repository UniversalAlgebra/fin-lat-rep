%% wheatstone_bridge.tex
%% "Wheatstone Bridge"
\node[lat] (x) at (-1,1) {};
\node[lat] (y) at (1,1) {};
\node[lat] (a) at (0,2) {};
\node[lat] (b) at (0,0) {};
\draw (x) node [left] {$x$};
\draw (a) node [above] {$a$};
\draw (b) node [below] {$b$};
\draw (y) node [right] {$y$};
\draw[semithick] (b)-- (a) node[pos=.5,right] {$\beta$};
\draw[semithick] (x)-- (a) node[pos=.5,left] {$\alpha_1$};
\draw[semithick] (x)-- (b) node[pos=.5,left] {$\alpha_2$};
\draw[semithick] (b)-- (y) node[pos=.5,right] {$\alpha_1$};
\draw[semithick] (a)-- (y) node[pos=.5,right] {$\alpha_2$};
%%   \caption{The Wheatstone Bridge which defines the relation 
%%     $\tau(\alpha_1, \alpha_2, \beta)$ as follows: 
%%     $(x,y) \in \tau(\alpha_1, \alpha_2, \beta)$ if and only if there 
%%     exist $a, b \in X$ satisfying the relations in the diagram.}

