
      \node (x) at (-1,1)  [draw, circle, inner sep=\dotsize] {};
      \node (y) at (1,1)  [draw, circle, inner sep=\dotsize] {};
      \node (a) at (0,2)  [draw, circle, inner sep=\dotsize] {};
      \node (b) at (0,0)  [draw, circle, inner sep=\dotsize] {};
      \draw (x) node [left] {$x$};
      \draw (a) node [above] {$a$};
      \draw (b) node [below] {$b$};
      \draw (y) node [right] {$y$};
      \draw[semithick] (b)-- (a) node[pos=.5,right] {$\beta$};
      \draw[semithick] (x)-- (a) node[pos=.5,left] {$\alpha_1$};
      \draw[semithick] (x)-- (b) node[pos=.5,left] {$\alpha_2$};
      \draw[semithick] (b)-- (y) node[pos=.5,right] {$\alpha_1$};
      \draw[semithick] (a)-- (y) node[pos=.5,right] {$\alpha_2$};
%%   \caption{The Wheatstone Bridge which defines the relation 
%%     $\tau(\alpha_1, \alpha_2, \beta)$ as follows: 
%%     $(x,y) \in \tau(\alpha_1, \alpha_2, \beta)$ if and only if there 
%%     exist $a, b \in X$ satisfying the relations in the diagram.}
%%   \label{fig:rhoagain}
%% \end{figure}
