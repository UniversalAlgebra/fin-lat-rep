

%%%%%%%%%%%%%%%%%%%%% TRANSITIVEGSETS %%%%%%%%%%%%%%%%%%%%%%%%%5
Let $X$ be a finite set and consider the set $X^X$ of all maps from $X$ to
itself, which, when endowed with composition of maps and the identity mapping,
forms a monoid, $\<X^X, \circ, \id_X\>$.  The submonoid $S_X$ of all bijective
maps in $X^X$ is a group, the \defn{symmetric group on} $X$.  When the
underlying set is more complicated, or for emphasis, we denote the symmetric
group on $X$ by $\Sym(X)$.  When the  
underlying set isn't important, we usually write $S_n$ to denote the
symmetric group on an $n$-element set. 

If we have defined some set $F$ of basic operations on $X$, so that
$\bX = \<X, F\>$ is an algebra, then two other important submonoids of
$X^X$ are $\End \bX$, the set of maps in $X^X$ which respect all 
operations in $F$, and $\Aut \bX$, the set of bijective maps in  $X^X$ which
respect all operations in $F$.  It is apparent from the definition that
$\Aut\bX= S_X \cap \End \bX$, and  $\Aut\bX$ is a submonoid of $\End \bX$
and a subgroup of $S_X$.  These four fundamental monoids associated with the
algebra $\bX$, and their relative ordering under inclusion, are shown in the diagram
below. 

\begin{center}
  \begin{tikzpicture}[scale=.7]
    \node (Aut) at (0,0.2) [draw,circle,inner sep=1pt] {};
    \draw[font=\small] (0,-.30) node {$\Aut\bX$};

    \node (End) at (-1.6,2) [draw,circle,inner sep=1pt] {};
    \draw[font=\small] (-2.6,2) node {$\End \bX$};

    \node (Sx) at (1.6,2) [draw,circle,inner sep=1pt] {};
    \draw (2.2,2) node {$S_X$};

    \node (XX) at (0,3.8) [draw,circle,inner sep=1pt] {};
    \draw (0,4.2) node {$X^X$};
    \draw[semithick,dotted]    (Aut) to (End) to (XX) to (Sx) to (Aut);
  \end{tikzpicture}
\end{center}



Given a finite group $G$, and an algebra $\bX = \<X, F\>$, a
\index{representation!of a finite group}%
\emph{representation} of $G$ on $\bX$ is a group homomorphism
from $G$ into $\Aut\bX$.  That is, a representation of $G$ is a mapping
$\varphi : G \rightarrow \Aut\bX$ which satisfies $\varphi(g_1 g_2) =
\varphi(g_1) \circ \varphi(g_2)$, where (as above) $\circ$ denotes composition
of maps in $\Aut\bX$.

\subsection{Transitive $G$-sets}
A representation $\varphi : G \rightarrow \Aut\bX$ defines an action by $G$ on the
set $X$, as follows: $\phi(g): x \mapsto x^{\varphi(g)}$.  If $\varphi(G) \leq
\Aut\bX$
denotes the image of $G$ under $\varphi$, we call the algebra $\< X, \phi(G)\>$
a \defn{G-set}.
% \footnote{More 
% generally, a \Gset\ is sometimes defined to be a pair $(X, \varphi)$, where
% $\varphi$ is a homomorphism from a group into the symmetric group $S_X$, see
% e.g.~\cite{Suzuki:1982}.}   
The action is called
\index{transitive!action}%
\emph{transitive} if for each pair $x, y \in X$ there is some $g\in
G$ such that $x^{\phi(g)} = y$. 
A group that acts transitively on some set is called a 
\index{transitive!group}%
\emph{transitive group}.
(Without specifying the set, however, this term is meaningless, since
every group acts transitively on some sets and intransitively on others.)
A representation $\varphi$ is called \emph{transitive} if the resulting action
is transitive. 


A representation  $\varphi : G \rightarrow \Aut\bX$ is called 
\emph{faithful}
\index{faithful!representation}%
if it is a monomorphism, in which case $G$ is isomorphic to its image under
$\varphi$, which is a subgroup of $\Aut\bX$.  We also say, in this case, that
the group $G$ acts faithfully, and call it a \defn{permutation group}.


The \defn{degree} of a group action on a set $X$ is the
cardinality of $X$.
Finally, a \defn{primitive group} is a group that
contains a core-free maximal subgroup.

For our purposes the most important representation of a group $G$ is its action 
on the set of cosets of a subgroup.  That is, for any subgroup $H\leq G$,
we define a transitive permutation representation of $G$, which we
will denote by $\rho_H$.  Specifically, $\rho_H$ is a group homomorphism
from $G$ into the symmetric group $\Sym(G/H)$ of permutations on the set $G/H =
\{H, Hx_1, Hx_2, \dots \}$ of \emph{right} cosets of $H$ in $G$.  

When 
The action is simply right multiplication by elements of $G$. That is,
$(Hx)^{\rho(g)}= Hxg$.
% Clearly, $\rho(g_1 g_2) = \rho(g_1)\rho(g_2)$ for all $g_1,
% g_2 \in G$, so $\rho$ is a group homomorphism.
Each $Hx$ is a point in the set $G/H$, and the
\defn{point stabilizer} of $Hx$ in $G$ is defined by
$G_{Hx} = \{g\in G \mid Hxg = Hx \}$.  Notice that 
$G_H = \{g\in G \mid Hg = H \} = H$ is the point stabilizer of $H$ in $G$, and 
\[
G_{Hx} =\{g\in G \mid Hxgx^{-1}  = H \} = 
x^{-1} G_H x  = x^{-1} H x = H^x.
\]
Thus, the kernel of the homomorphism $\rho$ is 
\[
\ker \rho = \{g\in G \mid \forall x \in G,\; Hxg = Hx \} = 
\bigcap_{x\in G}G_{Hx} = \bigcap_{x\in G} x^{-1} H x  = \bigcap_{x\in G} H^x.
\]
Note that $\ker \rho$ is the largest normal subgroup of $G$ 
contained in $H$, also known as the \defn{core} of $H$ in $G$, which we denote
by $\core_G(H)$.

If the subgroup $H$ happens to be \defn{core-free}, that is,
$\core_G(H)=1$, 
then $\rho : G \hookrightarrow \Sym(G/H)$, an embedding, so 
$\rho$ is a faithful representation; hence $G$ is a permutation group.



%%%%%%%%%%%%%%%%%%%%%%%% GSETISOMORPHISMS %%%%%%%%%%%%%%%%%%%%%%%%%%%%
% \subsection{$G$-set isomorphism theorems}
% \label{subsec:g-set-isomorphism}
% We have seen above that the action of a group on cosets of a subgroup $H$ is a
% transitive permutation representation, and the representation is faithful when
% $H$ is core-free. 
% The first theorem in this section states that every
% transitive permutation representation is of this form.
% (In fact, as we will see in Lemma~\ref{lem:intransitive-gsets} below, every permutation
% representation, whether transitive or not, can be viewed as an action on cosets.)

% First, we need some more notation. Given a \Gset\ $\bA = \<A, G\>$ and any
% element $a\in A$,  the set  
% $G_a = \{g\in G \mid ga = a\}$ 
% of all elements of $G$ which fix $a$ is a
% \index{stabilizer subgroup}
% subgroup of $G$, called the \emph{stabilizer of $a$ in $G$}.

% \begin{theorem}[1st \Gset\ Isomorphism Theorem]
% \label{thm:g-set-isomorphism1}
%   If $\bA = \<A, \barG\>$ is a transitive \Gset, then $\bA$ is
%   isomorphic to the \Gset 
%   \[
%   \Gamma :=  \<G/\stab{a}, \{\hat{\lambda}_g : g\in G\}\>
%   \]
%   for any $a\in A$.
% \end{theorem}
% \begin{proof}
%   Suppose $\bA= \<A, \barG\>$ is a transitive \Gset, so 
%   $A = \{\barg a \mid  g\in G\}$ for any $a\in A$.  The operations of the \Gset\ $\Gamma$ are defined, for each $g\in G$ and each
%   coset $x \stab{a}\in G/\stab{a}$, by $\hat{\lambda}_g(x \stab{a}) = gx\stab{a}$. 

%   Let 
%   $\bG_\Lambda$ denote the \Gset\ $\<G, \{\lambda_g : g\in G\}\>$, that is,
%   the group $G$ acting on itself by left multiplication. 
%   Fix $a\in A$, and define $\varphi_a:G \rightarrow A$ by $\varphi_a(x) =
%   \barx(a)$ for 
%   each $x\in G$.  
%   Then $\varphi_a$ is a homomorphism
%   from 
%   $\bG_\Lambda$ 
%   into $\bA$ --
%   that is, $\varphi_a$ respects operations:\footnote{In general, if $\bA
%     =\< A, F\>$ and $\bB = \<B, F\>$ are two algebras of the same
%     similarity type, then   
%     $\varphi: \bA \rightarrow \bB$ is a homomorphism provided
%     \[
%     \varphi(f^\bA(a_1,\dots, a_n)) = f^\bB(\varphi(a_1),\dots,\varphi(a_n))
%     \]
%     whenever $f^\bA$ is an $n$-ary operation of $\bA$, $f^\bB$ is the
%     corresponding $n$-ary operation of $\bB$, and $a_1,\dots, a_n$ are arbitrary 
%     elements of $A$.  (Note that a one-to-one correspondence between the
%     operations of 
%     two algebras of the same similarity type is assumed, and required for the
%     definition 
%     of homomorphism to make sense.)
%   }
%   \[
%   \varphi_a(\lambda_g(x)) = \varphi_a(gx)= \overline{gx}(a)
%   = \barg \cdot \barx(a)
%   = \barg \varphi_a(x).
%   \]
%   Moreover, since $\bA$ is transitive, $\varphi_a(G) = \{\barg a \mid  g\in G\}
%   = A$, 
%   so $\varphi_a$ is an epimorphism.  Therefore,
%   $\bG_\Lambda/\ker \varphi_a \cong \bA$.  
%   To complete the proof, one simply checks that the two algebras $\bG_\Lambda
%   /\ker \varphi_a $ and $\Gamma$ are identical.\footnote{
%     Indeed, 
%     $\ker \varphi_a = \{(x,y) \in G^2 \mid  \varphi_a(x) = \varphi_a(y)\}$
%     and the  universe of $\bG_\Lambda /\ker \varphi_a$ is 
%     $G/\ker \varphi_a = \{x/\ker \varphi_a  \mid x\in G \}$. 
%      where for each $x\in G$
%   \begin{align*}
%     x/\ker \varphi_a &= \{y\in G  \mid  (x,y) \in \ker \varphi_a\}
%     = \{y\in G  \mid  \varphi_a(x) = \varphi_a(y)\}
%     = \{y\in G  \mid  \barx(a) = \bary(a)\}\\
%     &= \{y\in G  \mid  \id_{A}(a) = \overline{x^{-1}y}(a)\}
%     = \{y\in G  \mid  x^{-1}y \in \stab{a}\}
%     = x \stab{a}.
%   \end{align*}
%   These are precisely the elements of $G/\stab{a}$,
%   so the universes of $\bG_\Lambda /\ker \varphi_a$ and $\Gamma$ are
%   the same, as are their operations (left multiplication by $g\in G$).}
% \end{proof}

% The next theorem shows why intervals of subgroup lattices are so important for
% our work.
\begin{theorem}[\Gset\ Isomorphism Theorem]
  \label{thm:g-set-isomorphism2}
  Let $\bA = \<A, G\>$ be a transitive \Gset\ and fix $a\in A$.  
  Then the lattice $\Con \bA$ is isomorphic to the
  interval $\lb G_a, G\rb$ in the subgroup lattice of $G$.
\end{theorem}
% \begin{proof}
%     For each $\theta  \in \Con \bA$, let $H_\theta =\{g\in G \mid (g(a),a) \in
%     \theta\}$,  
%     and for each $H \in [G_a, G]$, let 
%     $(b, c) \in \theta_H$ mean there exist $g \in G$ and $h \in H$ 
%     such that $gh(a) = b$ and $g(a) =c$. 
%     If $g_1, g_2 \in H_\theta$, then 
%     \[
%     (g_2(a), a) \in \theta \quad \Rightarrow \quad (g_2^{-1} g_2(a), g_2^{-1}(a)) =
%     (a, g_2^{-1}(a)) \in \theta,
%     \]
%     so $(g_2^{-1}(a), a) \in \theta$, by symmetry.  Therefore,
%     $(g_1g_2^{-1} (a), g_1 (a)) \in \theta$, so 
%     $(g_1g_2^{-1} (a), (a)) \in \theta$, by transitivity. Thus $H_\theta$ is a
%     subgroup of 
%     $G$, and clearly  $G_a \leq H_\theta$. 
%     It is also easy to see that $\theta_H$ is a congruence of $\bA$.
%     The equality $H_{\theta_H}=H$ trivially follows from the definitions. 
%     On the other hand $(b, c) \in \theta_{H_\theta}$ if and only if there exist 
%     $g, h\in G$ for which $(h(a),a) \in \theta$ and
%     $b = gh(a)$, and $c = g(a)$. Since $G$ is transitive, it is equivalent 
%     to $(b, c) \in \theta$.
%     Therefore, $\theta_{H_\theta} = \theta$. Finally, $H_\theta \leq H_\phi$ if
%     and only if
%     $\theta \leq \phi$, so $\theta \mapsto H_\theta$ is an isomorphism
%     between $\Con \bA$ and $[G_a, G]$.
% \end{proof}

Since the foregoing theorem is so central to our work, we provide an alternative 
statement of it. This is the version typically found in group theory 
textbooks (e.g., \cite{Dixon:1996}).  Keeping these two alternative perspectives in
mind can be useful.

\begin{theorem}[\Gset\ Isomorphism Theorem, version 2]
  Let $\bA = \<A, \phi(G)\>$ be a transitive \Gset\
  and fix $a \in A$. Let $\sB$ be the set of all blocks $B$ that contain $a$.
  % Let $[\stab{a},G] \subseteq \Sub(G)$ denote the set of all subgroups of 
  % $G$ containing $\stab{a}$. 
  Then there is a bijection $\Psi :\sB \rightarrow \lb \stab{a},G\rb$ given by $\Psi(B)= G(B)$,
  with inverse mapping $\Phi: \lb \stab{a},G\rb \rightarrow \sB $ 
  given by $\Phi(H) = \{a^{\phi(h)}  \mid  h\in H\}$. 
  The mapping $\Psi$ is order-preserving in the sense
  that
  $B_1\subseteq B_2 \Leftrightarrow \Psi(B_1) \leq \Psi(B_2)$.
\end{theorem}

Briefly, the poset $\<\sB, \subseteq\>$ is order-isomorphic to the 
poset $\<[\stab{a},G], \leq\>$. 

\begin{corollary}
  Let $G$ act transitively on a set with at least two
  points. 
  Then $G$ is primitive if and only if each stabilizer $\stab{a}$ is a
  maximal subgroup of $G$.
\end{corollary}

Since the point stabilizers of a transitive group are all conjugate, 
one stabilizer is maximal only when all of the stabilizers are maximal. 
In particular, a regular permutation group is primitive if and only if it has
prime degree. 

Next we describe (up to equivalence) all transitive permutation
representations of a given group $G$.  
We call two representations (or actions) 
\index{equivalent representations}%
\emph{equivalent}
provided the associated $G$-sets are isomorphic. 
The foregoing implies that every transitive permutation representation of $G$ is
equivalent to $\hlambda_H$ for some subgroup $H \leq G$.  The following
lemma\footnote{Lemma 1.6B of \cite{Dixon:1996}.} 
shows that we need only consider a single representative $H$ from each of the
conjugacy classes of subgroups.  

\begin{lemma}
  Suppose $G$ acts transitively on two sets,
  $A$ and $B$.  Fix $a\in A$ and let $G_a$ be the stabilizer of $a$ (under the first
  action).  Then the two actions are equivalent
  if and only if the subgroup $G_a$ is also a stabilizer under the second action
  of some point $b\in B$. 
\end{lemma}

The point stabilizers of the action $\hlambda_H$ described above are the
conjugates of $H$ in $G$.  Therefore, the lemma implies that, for any two
subgroups $H, K \leq G$, the representations $\hlambda_H$ and $\hlambda_K$ are
equivalent precisely when $K = x Hx^{-1}$ for some $x\in G$. 
Hence, the transitive permutation representations of $G$ are given, up to
equivalence, by $\hlambda_{K_i}$ as $K_i$ runs over a set of representatives of
conjugacy classes of subgroups of $G$.   
