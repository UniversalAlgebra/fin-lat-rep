
In this section, we introduce a strategy that has proven very useful for showing
that a given lattice is representable as a congruence lattice of a finite
algebra. We call it the \defn{closure method}, and it has become especially
useful with the advent of powerful computers which can search for such
representations.  Here,  $\Eq X$ denotes the lattice of equivalence
relations on $X$. Sometimes we abuse notation and take $\Eq X$ to mean the
lattice of partitions of the set $X$. This never causes problems because these
two lattices are isomorphic.  

\subsection{Concrete versus abstract representations}
As Bjarni J\'onsson explains in~\cite{Jonsson:1972}, there are two types of
representation problems for congruence lattices, the concrete and the
abstract.  The \emph{concrete representation problem} asks whether a specific family of
equivalence relations on a set $A$ is equal to $\Con \bA$ for some
algebra $\bA$ with universe $A$.  The \defn{abstract representation problem}
asks whether a given lattice is isomorphic to $\Con \bA$ for some algebra $\bA$.

These two problems are closely related, and have become even more so since the
publication in 1980 
of~\cite{Pudlak:1980}, in which Pavel \Pudlak\ and
\Jiri\ \Tuma\ prove that every finite
lattice can be embedded as a 
spanning sublattice\footnote{Recall, by a 
  \defn{spanning sublattice}
  of a bounded lattice $L_0$, we mean a sublattice $L\leq L_0$ that has the same top and
  bottom as $L_0$.  That is  $1_L = 1_{L_0}$ and $0_L = 0_{L_0}$.}
of the lattice $\Eq X$ of equivalence relations on a finite set $X$.   
Given this result, we see that even if our goal is to solve the abstract
representation problem for some (abstract) lattice $L$, then
we can embed $L$ into $\Eq X$ as $L\cong L_0\leq \Eq X$, for some finite set
$X$, and then try to solve the concrete representation problem for $L_0$.  

A point of clarification is in order here.  The term 
\defn{representation} 
has become a bit overused in the literature about the finite lattice
representation problem.  On the one hand, given a finite lattice $L$, if there
is a finite algebra $\bA$ such that $L \cong \Con \bA$, then $L$ is called a
\defn{representable lattice}.  On the other hand, given a sublattice $L_0\leq \Eq X$, 
if $L_0\cong L$, then $L_0$ is sometimes called a 
\defn{concrete representation}
of the lattice $L$ (whether or not it is the congruence lattice of an algebra).    
Below we will define the notion of a \defn{closed concrete representation}, and if we
have this special kind of concrete representation of a give lattice, then that
lattice is indeed representable in the first sense.

As we will see below, there are many examples in which a particular concrete
representation $L_0\leq \Eq X$ of $L$ is not a congruence lattice of a
finite algebra.  (In fact, we will describe general situations in which we can
guarantee that there are no non-trivial\footnote{By a 
  \defn{non-trivial function} we mean a function that is
  not constant and not the identity.} operations which respect the equivalence
relations of $L_0$.)  This does not imply that $L \notin \sL$.  It may
simply mean that $L_0$ is not the ``right'' concrete representation of $L$, and
perhaps we can find some other $L \cong L_1\leq \Eq X$ such that $L_1 = \Con
\<X, \lambda(L_1)\>$.

\subsection{The closure method}
\label{sec:closure-method}
The idea described in this section
first appeared in \emph{Topics in Universal Algebra}~\cite{Jonsson:1972}, pages
174--175, where J\'onsson states, ``these or related results were discovered
independently by at least three different parties during the summer and fall of
1970: by Stanley Burris, Henry Crapo, Alan Day, Dennis Higgs and Warren Nickols
at the University of Waterloo, by R.~Quackenbush and B.~Wolk at the University
of Manitoba, and by B.~J\'{o}nsson at Vanderbilt University.''

Let $X^X$ denote the set of all (unary) maps from the set $X$ to itself, and let 
$\Eq X$ denote the lattice of equivalence relations on the set $X$.  If $\theta
\in \Eq X$ and $h\in X^X$, we write $h(\theta) \subseteq \theta$ and say
that ``$h$ respects $\theta$'' if and only if for all $(x,y)\in X^2$ $(x,y)\in
\theta$ implies 
$(h(x),h(y)) \in \theta$.  If $h(\theta) \nsubseteq \theta$, we sometimes say
that ``$h$ violates $\theta$.''

For $L\subseteq \Eq X$ define
\[
\lambda(L) = \{h\in X^X: (\forall \theta \in L) \; h(\theta) \subseteq \theta \}.
\]

For $H\subseteq X^X$ define
\[
\rho(H) = \{\theta \in \Eq X \mid   (\forall h\in H) \; h(\theta) \subseteq \theta\}.
\]
The map $\rho \lambda$ is a \defn{closure operator} on $\Sub[\Eq X]$.
That is, $\rho \lambda$ is
\begin{itemize}
\item \emph{idempotent:}\footnote{In fact, $\rho \lambda \rho = \rho$ and 
  $\lambda \rho \lambda = \lambda$.} $\rho \lambda \rho \lambda = \rho \lambda$;
\item \emph{extensive:} $L \subseteq \rho \lambda (L)$ for every $L \leq \Eq X$;
\item \emph{order preserving:} $\rho \lambda (L) \leq \rho \lambda (L_0)$ if $L \leq L_0$.
\end{itemize}
Given $L\leq \Eq X$, if $\rho\lambda(L) = L$, then we say $L$ is a 
%\defn{closed sublattice of $\Eq X$}, in which case we clearly have
\emph{closed} sublattice of $\Eq X$, in which case we clearly have
\[L = \Con \<X, \lambda(L)\>.\]
This suggests the following strategy for solving the representation problem for a
given abstract finite lattice $L$: search for a concrete representation $L \cong
L_0\leq \Eq X$,
compute $\lambda(L_0)$, compute $\rho\lambda(L_0)$, and determine whether 
$\rho\lambda(L_0) = L_0$.  If so, then we have solved the abstract representation
problem for $L$, by finding a \defn{closed concrete representation}, or simply
\emph{closed representation}, of $L_0$.  We call this strategy the \defn{closure method}.

We now state without proof a well known theorem which shows that the finite lattice
representation problem can be formulated in terms of closed concrete
representations (cf.~\cite{Jonsson:1972}).
\begin{theorem}\label{Concrete-thm-3}
  If $\bL \leq \bEqX$, then $\bL= \bCon\bA$ for some algebra 
  $\mathbf{A} = \langle X, F\rangle$ if and only if $\bL$ is closed.
\end{theorem}

%% In the remaining sections of this chapter, we consider various aspects of the
%% closure method and prove some results about it.  Later, in
%% Section~\ref{sec:seven-elem-latt}, we apply it to the problem of finding
%% closed representations of all lattices of small order.  
Before proceeding, we introduce a slightly different set-up than the
one introduced above that we have found particularly useful
for implementing the closure method on a computer. Instead of considering the
set of equivalence relations on a finite set, we work with the set of idempotent
decreasing maps.  These were introduced above in
Section~\ref{sec:duals-interv-subl-detail}, but we briefly review the definitions here
for convenience.

Given a totally ordered set $X$, 
%consider the subset of the set $X^X$ of all functions that map $X$ into itself.  
let the set $\idemdec = \{f\in X^X: f^2 = f \text{ and } f(x) \leq x\}$ be partially
ordered by $\sqsubseteq$ as follows:
\[
f\sqsubseteq g \quad \Leftrightarrow \quad \ker f \leq \ker g.  
\]
As noted above, 
this makes \idemdec\ into a lattice that is isomorphic to $\bEqX$.   
Define a relation $R$ on $X^X \times \idemdec$ as follows: 
\[
(h,f) \in R \quad \Leftrightarrow \quad (\forall (x,y) \in \ker f)\; (h(x),h(y))
\in \ker f.
\]
If $h\, R\, f$, we say that  $h$ \emph{respects} $f$.

Let $\sF = \power{\idemdec}$ and $\sH = \power{X^X}$ be partially ordered by set
inclusion, and define the maps 
$\lambda: \sF \rightarrow \sH$ and $\rho: \sH \rightarrow \sF$ as follows:
\[
\lambda(F) = \{h\in X^X: \forall f\in F,\, h\, R \,f\} \quad (F \in \sF)
\]
\[
\rho(H) = \{f\in \idemdec: \forall h\in H,\, h\, R\, f\} \quad (H \in \sH)
\]
The pair $(\lambda, \rho)$ defines a \defn{Galois correspondence} between
$\idemdec$ and $X^X$.  That is, $\lambda$ and $\rho$ are 
antitone %(order-reversing) 
maps such that $\lambda \rho \geq \id_{\sH}$ and $\rho
\lambda \geq \id_{\sF}$.  In particular, for any set $F \in \sF$ we have 
$F \subseteq \rho \lambda (F)$.  These statements are all trivial verifications, and
a couple of easy consequences are:
\begin{enumerate}
\item  $\rho\lambda\rho = \rho$ and $\lambda\rho \lambda= \lambda$,
\item $\rho \lambda$ and $\lambda \rho$ are idempotent.
\end{enumerate}

Since the map $\rho \lambda$ from $\sF$ to itself is idempotent, extensive, %($F \subseteq \rho \lambda (F)$) 
and order preserving, it is a 
\defn{closure operator} 
on $\sF$, and we say a set $F\in \sF$ is
\emph{closed} if and only if $\rho\lambda(F) = F$. Equivalently,
$F$ is closed if and only if $F = \rho(H)$ for some $H\in \sH$.
