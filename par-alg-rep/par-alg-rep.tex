%%%%%%%%%%%%%%%%%%%%%%%%%%%%%%%%%%%%%%%%%%%%%%%%%%%%%%%%%%%%%%%%%%%%%%%%%%%%%%%
%%                         BIBLIOGRAPHY FILE                                 %%
%%%%%%%%%%%%%%%%%%%%%%%%%%%%%%%%%%%%%%%%%%%%%%%%%%%%%%%%%%%%%%%%%%%%%%%%%%%%%%%
%% The `filecontents` command will crete a file in the inputs directory called 
%% refs.bib containing the references in the document, in case this file does 
%% not exist already.
%% If you want to add a BibTeX entry, please don't add it directly to the
%% refs.bib file.  Instead, add it in this file between the
%% \begin{filecontents*}{refs.bib} and \end{filecontents*} lines
%% then delete the existing refs.bib file so it will be automatically generated 
%% again with your new entry the next time you run pdfaltex.
\begin{filecontents*}{inputs/refs.bib}
@misc{Lampe:20161017,
  author        = "Bill Lampe",
  howpublished  = "personal communication",
  note          = "October 17",
  year          = "2016"
}
@article {MR552159,
    AUTHOR = {Pudl{\'a}k, Pavel and T{\.u}ma, Ji{\v{r}}{\'{\i}}},
     TITLE = {Every finite lattice can be embedded in a finite partition
              lattice},
   JOURNAL = {Algebra Universalis},
  FJOURNAL = {Algebra Universalis},
    VOLUME = {10},
      YEAR = {1980},
    NUMBER = {1},
     PAGES = {74--95},
      ISSN = {0002-5240},
   MRCLASS = {06B15 (05C99)},
  MRNUMBER = {552159},
MRREVIEWER = {James W. Lea, Jr.},
       DOI = {10.1007/BF02482893},
       URL = {http://dx.doi.org/10.1007/BF02482893},
}
@article {MR3076179,
    AUTHOR = {Kearnes, Keith A. and Kiss, Emil W.},
     TITLE = {The shape of congruence lattices},
   JOURNAL = {Mem. Amer. Math. Soc.},
  FJOURNAL = {Memoirs of the American Mathematical Society},
    VOLUME = {222},
      YEAR = {2013},
    NUMBER = {1046},
     PAGES = {viii+169},
      ISSN = {0065-9266},
      ISBN = {978-0-8218-8323-5},
   MRCLASS = {08B05 (08B10)},
  MRNUMBER = {3076179},
MRREVIEWER = {James B. Nation},
       DOI = {10.1090/S0065-9266-2012-00667-8},
       URL = {http://dx.doi.org/10.1090/S0065-9266-2012-00667-8},
}
@unpublished{Nation-notes,
author = {J. B. Nation},
title = {Notes on Lattice Theory},
note = {Unpublished notes},
year = {2007, 2016},
URL = {http://www.math.hawaii.edu/~jb/}
}
\end{filecontents*}
\documentclass[12pt]{amsart}
% The following \documentclass options may be useful:
% preprint      Remove this option only once the paper is in final form.
% 10pt          To set in 10-point type instead of 9-point.
% 11pt          To set in 11-point type instead of 9-point.
% numbers       To obtain numeric citation style instead of author/year.

%% \usepackage{setspace}\onehalfspacing

\usepackage{amsmath}
\usepackage{amscd,amssymb,amsthm} %, amsmath are included by default
\usepackage{latexsym,stmaryrd,mathrsfs,enumerate,scalefnt,ifthen}
\usepackage{mathtools}
\usepackage[mathcal]{euscript}
\usepackage[colorlinks=true,urlcolor=black,linkcolor=black,citecolor=black]{hyperref}
\usepackage{url}
\usepackage{scalefnt}
\usepackage{tikz}
\usepackage{color}
\usepackage[margin=1in]{geometry}
\usepackage{scrextend}

%%////////////////////////////////////////////////////////////////////////////////
%% Theorem styles
\numberwithin{equation}{section}
\theoremstyle{plain}
\newtheorem{theorem}{Theorem}[section]
\newtheorem{lemma}[theorem]{Lemma}
\newtheorem{proposition}[theorem]{Proposition}
\newtheorem{prop}[theorem]{Proposition}
\theoremstyle{definition}
\newtheorem{claim}[theorem]{Claim}
\newtheorem{corollary}[theorem]{Corollary}
\newtheorem{definition}[theorem]{Definition}
\newtheorem{notation}[theorem]{Notation}
\newtheorem{Fact}[theorem]{Fact}
\newtheorem*{fact}{Fact}
\newtheorem{example}[theorem]{Example}
\newtheorem{examples}[theorem]{Examples}
\newtheorem{exercise}{Exercise}
\newtheorem*{lem}{Lemma}
\newtheorem*{cor}{Corollary}
\newtheorem*{remark}{Remark}
\newtheorem*{remarks}{Remarks}
\newtheorem*{obs}{Observation}


%%%%%%%%%%%%%%%%%%%%%%%%%%%%%%%%%%%%%%%%
% Acronyms
%%%%%%%%%%%%%%%%%%%%%%%%%%%%%%%%%%%%%%%%
%% \usepackage[acronym, shortcuts]{glossaries}
%\usepackage[smaller]{acro}
\usepackage[smaller]{acronym}
\usepackage{xspace}

%% \acs{CSP} -- short version of the acronym\\
%% \acl{CSP} -- expanded acronym without mentioning the acronym.\\
%% \acp{CSP} -- plurals.\\
%% \acfp{CSP} -- long forms into plurals.\\
%% \acsp{CSP} -- short form into a plural.\\
%% \aclp{CSP} -- long form into a plural.\\
%% \acfi{CSP} -- Full Name acronym in italics and abbreviated form in upshape.\\
%% \acsu{CSP} -- short form of the acronym and marks it as used.\\
%% \aclu{CSP} -- Prints the long form of the acronym and marks it as used.\\

\acrodef{lics}[LICS]{Logic in Computer Science}
\acrodef{sat}[SAT]{satisfiability}
\acrodef{nae}[NAE]{not-all-equal}
\acrodef{ctb}[CTB]{cube term blocker}
\acrodef{tct}[TCT]{tame congruence theory}
\acrodef{wnu}[WNU]{weak near-unanimity}
\acrodef{CSP}[CSP]{constraint satisfaction problem}
\acrodef{MAS}[MAS]{minimal absorbing subuniverse}
\acrodef{MA}[MA]{minimal absorbing}
\acrodef{cib}[CIB]{commutative idempotent binar}
\acrodef{sd}[SD]{semidistributive}
\acrodef{NP}[NP]{nondeterministic polynomial time}
\acrodef{P}[P]{polynomial time}
\acrodef{PeqNP}[P $ = $ NP]{P is NP}
\acrodef{PneqNP}[P $ \neq $ NP]{P is not NP}

%%%%%%%%%%%%%%%%%%%%%%%%%%%%%%%%%%%%%%%%%%%%%%%%%%%%%%%%%%%%%%%%%

\usepackage{inputs/proof-dashed}


%%%%%%%%%%%%%%%%%%%%%%%%%%%%%%%%%%%%%%%%%%%%%%%%%%%%%%%%%%%%%%%%%

%% Put new macros in the macros.sty file
\usepackage{inputs/macros}

\usepackage[backend=bibtex]{biblatex}
\bibliography{inputs/refs.bib}

\begin{document}

\title[Partial Algebras]{Every Finite lattice is the congruence lattice\\
of a Finite Partial Algebra}
\date{\today}
%% \author[W.~DeMeo]{William DeMeo}
\address{University of Hawaii}
\email{williamdemeo@gmail.com}
\address{Chapman University}
\email{jipsen@chapman.edu}

%% \thanks{The authors would like to extend special thanks to...}

\maketitle

%% \begin{abstract}\end{abstract}

\section{Introduction}
\label{sec:introduction}
This note begins with a proof in Section~\ref{sec:simple-proof-well} of the
result stated in the title. Bill Lampe~\cite{Lampe:20161017} pointed out that this result has been
known for a long time and subsequently explained to DeMeo a more general
viewpoint.  In Section~\ref{sec:more-gener-appr}
we give some background about closure operators and then
reiterate what Lampe explained. The presentation is imperfect, but
hopefully it will elicit comments and criticisms that we can use to improve it.

\section{A well known result}
\label{sec:simple-proof-well}
In this section we give a straight-forward proof of the fact that every finite
lattice is the congruence lattice of a finite partial algebra.


\begin{lemma}
Let $X$ be a finite set, and let $\Eq(X)$ denote the lattice of equivalence
relations on $X$. If $L\leq \Eq(X)$ is a 0-1-sublattice,
and $\rho \in \Eq(X)$ and $\rho \notin L$, then for some $k < \omega$ there exists a partial
operation $f\colon X^k \rightharpoonup X$ that is compatible with $L$ and
incompatible with $\rho$.
\end{lemma}
\begin{proof}
  First we focus on the relations in $L$ that are above $\rho$.
  Let $\rho^\uparrow \cap L = \{\gamma \in L \mid \gamma \geq \rho\}$.
  Since $\rho\notin L$, we have $\gamma > \rho$ for all $\gamma \in \rho^\uparrow \cap L$.
  Now, $\rho^\uparrow \cap L$ has a least element
  $\rho^* = \Meet (\rho^\uparrow \cap L)$.  Clearly
  $\rho^*\geq \rho$ and since $\rho^* \in L$ we have
  $\rho^*\neq \rho$, so
  $\rho^* > \rho$.  Therefore, there exists $(u,v) \in \rho^* - \rho$.

  Next consider the elements of $L$ that are not above $\rho$. For each such
  $\alpha_i \in L - \rho^\uparrow$ there exists $(x_i, y_i) \in \rho -\alpha_i$.
  Let $(x_1, y_1), \dots, (x_k, y_k)$ be the list of all unique such pairs
  (i.e., each pair appears in the list exactly once).
  Define the partial function $f\colon X^k \rightharpoonup X$ at only two points of $X^k$; specifically, let
  \[ f(x_1, \dots, x_k) = u \quad \text{ and } \quad f(y_1, \dots, y_k) = v. \]
  Then, since $(\forall i)(x_i, y_i) \in \rho$ and $(u,v) \notin \rho$, 
  $f$ is incompatible with $\rho$.  On the other hand,
  $(u,v) \in  \rho^* = \Meet (\rho^\uparrow \cap L)$, so
  $(u,v) \in \gamma$  for every $\gamma \in \rho^\uparrow \cap L$, so
  $f$ is compatible with every $\gamma \in \rho^\uparrow \cap L$.
  
  
  Finally, for each $\alpha_i\in L$ not above $\rho$ there is at least one pair
  $(x_i, y_i)\notin \alpha_i$.  Therefore, it is impossible for $f$ to be
  incompatible with any such $\alpha_i$. 
\end{proof}

\begin{theorem}
Let $X$ be a finite set and let $L\leq \Eq(X)$ be a 0-1-sublattice.
Then there exists a finite partial algebra
$\mathbb X = \< X, F\>$ with $\Con(\mathbb X) =  L$.
\end{theorem}

\begin{proof}
  By the lemma, for each $\rho \in \Eq(X) - L$, there exists $k< \omega$ and
  $f_\rho \colon  X^k \rightharpoonup X$ such that $f_\rho$ is compatible with every relation in
  $L$ and incompatible with $\rho$.  Let $\mathcal{R}$ be the set $\Eq(X) - L$ of
  all equivalence relations on $X$ that do not belong to $L$.  Define,
  $F = \{f_\rho \mid \rho \in \mathcal{R}\}$.  Evidently, $\Con \<X, F\> = L$.
\end{proof}

\section{Generalities}
\label{sec:more-gener-appr}

\subsection{Closure systems, closure operators, and Moore families}
First we recall some standard definitions.
%% \footnote{See J. B. Nation's notes~\cite{Nation-notes} for more details.}
A \defn{closure system} on a set $X$ is a collection $\sC$
of subsets of $X$ that is closed under arbitrary intersection (including the empty 
intersection, so $\bigcap \emptyset = X \in \sC$). 
Thus a closure system is a complete meet semilattice with respect to subset
inclusion ordering. 
Since every complete meet semilattice is automatically a complete lattice
(see \cite[Theorem 2.5]{Nation-notes}), 
the closed sets of a closure system form a complete lattice. 
%% The sets in $\sC$ are called \defn{closed sets}. 

Examples of closure systems that are especially relevant for our work are the following:
\begin{itemize}
\item order ideals of an ordered set
\item subalgebras of an algebra 
\item equivalence relations on a set
\item congruence relations of an algebra
\end{itemize}

\newcommand{\cl}{\ensuremath{\operatorname{\sansC}}}

Let $\bP = \<P, \leq \>$ be a poset.
An function $\cl \colon P \to P$ is called a \defn{closure operator} on $\bP$
if it satisfies the following axioms for all $x, y\in P$.
\begin{enumerate}
\item $x \leq \cl x$ (extensivity) 
\item $x \leq y$ implies $\cl x \leq \cl y$ (monotonicity) 
\item $\cl \cl x = \cl x$ (idempotence) 
\end{enumerate}
Thus, a closure operator is an extensive idempotent poset endomorphism,
and the definition above is equivalent to the single axiom
$(\forall\, x, y \in P) (x \leq \cl y \, \longleftrightarrow  \, \cl x \leq  \cl y)$.

\begin{example}
Let $X$ be a set and let $Y \subseteq X$.
Define $\cl_Y\colon \sP(X) \to \sP(X)$ by $\cl_Y (W) = W \cup Y$.
Then $\cl_Y$ is a closure operator on $\<\sP(X), \subseteq\>$.
\end{example}


A \defn{fixpoint} of a
function $\cl \colon  P \to P$ is an $x\in P$ satisfying $\cl x = x$.
A fixpoint of a closure operator is called \defn{closed}.
%% A \defn{closure operator} on a set $X$ is a map $\sansc  \colon \sP(X) \to \sP(X)$ 
%% satisfying, for all $A, B \in \sP(X)$,
%% \begin{enumerate}[(a)]
%% \item $A \subseteq \sansC A$ (extensivity) 
%% \item $A \subseteq B$ implies $\sansC A \subseteq \sansC B$ (monotonicity) 
%% \item $\sansC  \sansC A = \sansC A$ (idempotence) 
%% \end{enumerate}
%% A fixpoint of a closure operator is called a \defn{closed set}.
If the poset $\bP = \<P, \leq\>$ happens to be a complete lattice,
then by extensivity the largest element $\top = \Join P$ is
a fixpoint of every closure operator on $\bP$.
Also, the collection of fixpoints of a closure operator is closed under arbitrary meets.
(Proof: If $\sA$ is a set of fixpoints of $\cl$ and  
$a \in \sA$, then $\Meet \sA \leq a$, so by monotonicity
$\cl \bigl( \Meet \sA \bigr) \leq \cl a = a$. 
Since $a$ was arbitrary,
$\cl \bigl( \Meet \sA \bigr) \leq  \Meet \sA$.
By extensivity,
$\Meet \sA \leq \cl \bigl( \Meet \sA\bigr)$. % \subseteq \bigcap \sA$,
Therefore, $\cl \bigl( \Meet \sA \bigr) =  \Meet \sA$.)
%% ; that is,  $\Meet \sA$ is a fixpoint of $\cl$.)
The set of closure operators on $\bP$
themselves form a complete lattice under the pointwise
order: $\cl_1 \leq  \cl_2$ iff $\cl_1 x \leq  \cl_2 x$ for all $x \in P$. 

Some of these observations can be restated as follows:
if $\bP = \<P, \leq\>$ is a complete lattice,
then the set $\sC \subseteq P$ of fixedpoints of a closure operator
is a \defn{Moore family} on $\bP$---that is, 
%% A subset $\sM \subseteq P$ is called a \defn{Moore family} if
$\Join P \in \sC$ and every nonempty subset of $\sC$ is closed under arbitrary meets.
%% If $\bL = \<L, \join, \meet\>$ is a complete lattice with top $\top = \Join L$, then
%% That is, if $\emptyset \neq S \subseteq \sM$, then $\Meet S \in S$.
%% If $\bL = \<L, \join, \meet\>$ is a complete lattice, then a


Conversely, if we are given a Moore family $\sC$ on $\bP$, and if we define 
$\cl \colon P \to P$ by
\[
\cl a = \Meet \{b \in \sC \mid a\leq b\},
\]
then $\cl$ satisfies conditions (1)--(3)
above, making it a closure operator.
To summarize, $\sC \subseteq P$ is the set of fixpoints (i.e., closed elements)
of a closure operator on
$\bP$ if and only if $\sC$ is a Moore family on $\bP$.

Every Moore family on $\bP$ is itself the universe of a complete lattice with the order inherited
from $\bP$, though the join may differ from the join of $\bP$.

\subsubsection{More Moore families}
The name ``closure system'' is typically reserved for the special case
in which $\bP$ happens to be the powerset Boolean
algebra of a set $X$---that is, $\bP = \<\sP(X), \subseteq\>$; in that case, a Moore family on $\bP$ 
is called a closure system on $X$. 

\begin{example}
  Let $X$ be a set,
  let $\sansR(X) = \bigcup_{n<\omega}\sP(X^n)$ be the set of all finitary
  relations on $X$, and let
  %% let $\Eq(X)$ denote the lattice of equivalence relations on $X$, and let
  $\sansO(X) = \bigcup_{n< \omega} X^{X^n}$ be the set of all finitary operations on $X$.
  Define $F \colon  \sP(\sansR(X)) \to \sP(\sansO(X))$
  and $G \colon \sP(\sansO(X)) \to \sP(\sansR(X))$ as follows:
  if $A \subseteq\sansR(X)$ and $B \subseteq \sansO(X)$, then
  %% \sF_n(S) = \{f \in \sansO(X)X^{X^n} \mid f \text{ is compatible with every $s\in S$}\}
  %% \[  \sF(S) = \bigcup \sF_n(S)  \]
  \[
  F(A) = \{f \in \sansO(X) \mid f \text{ is compatible with every relation in $A$}\},
  \]
  \[
  G(B) = \{\rho \in \sansR(X) \mid \rho \text{ is compatible with every operation in $B$}\}.
  \]
  Then $G \circ F$ is a closure operator on the lattice % $\<\sansR(X), \subseteq\>$
  of all relations on $X$.
\end{example}

\begin{example}
  Let $X$ be a set,
  let $\Eq(X)$ denote the lattice of equivalence relations on $X$, and let
  $X^X$ be the set of all unary operations on $X$.
  Define $F_1 \colon  \sP(\Eq(X)) \to \sP(X^X)$
  and $G_1 \colon \sP(X^X) \to \sP(\Eq(X))$ as follows:
  if $A \subseteq\Eq(X)$ and $B \subseteq X^X$, then
  %% \sF_n(S) = \{f \in \sansO(X)X^{X^n} \mid f \text{ is compatible with every $s\in S$}\}
  %% \[  \sF(S) = \bigcup \sF_n(S)  \]
  \[
  F_1(A) = \{f \in X^X\mid f \text{ is compatible with every relation in $A$}\},
  \]
  \[
  G_1(B) = \{\rho \in \Eq(X) \mid \rho \text{ is compatible with every operation in $B$}\}.
  \]
  Then $G_1 \circ F_1$ is a closure operator on the lattice $\Eq(X)$
  of all equivalence relations on $X$.
\end{example}
\newcommand{\Luv}{\ensuremath{L_{u,v}}}
\newcommand{\bLuv}{\ensuremath{\mathbf{L}_{u,v}}}
\newcommand{\juv}{\ensuremath{\vee_{u,v}}}
\newcommand{\suv}{\ensuremath{\sigma_{u,v}}}
\begin{example}[Pudl{\'a}k-T{\.u}ma~\cite{MR552159}]
  Let $\bL = \<L, \join, \meet\>$ be a lattice, $u, v \in L$, and $u\leq v$.
  Define a subset $\Luv$ of $L$ by
  $\Luv =\{x\in L \mid v\leq x \text{ or } u \nleq x\}$.
  The partial order relation of the lattice $\bL$ induces a lattice order on
  $\Luv$. Denote the resulting lattice by $\bLuv$.
  Then the meet of $\bLuv$ is that of $\bL$, whereas the join of $\bLuv$ is
  \[
  x \juv y = 
  \begin{cases}
    x \join y, & \text{ if $u \nleq x \join y$,}\\
    x \join y \join v, & \text{ if $u \leq x \join y$.}
  \end{cases}
  \]
  Define a mapping $\suv \colon L \to \Luv$ as follows:
  \[
  \suv (x)  = 
  \begin{cases}
    x, & \text{ if $u \nleq x$,}\\
    x \join v, & \text{ if $u \leq x$.}
  \end{cases}
  \]
  Then $\suv$ is a surjective join-homomorphism. In fact, every
  join-homomorphism $\phi \colon L \to K$ satisfying $\phi(u) = \phi(v)$ splits as
  $\phi =  \psi \circ \suv$ for some $\psi \colon \Luv \to K$.
  (See the commutative diagram in Figure~\ref{fig:splitting}.)

  As a mapping from $\bL$ to itself, $\suv$ does not preserve joins. However,
  $\suv \colon L \to L$ is a closure operator and 
  $\Luv$ is the set of fixpoints of $\suv$ (the closed sets).

  \begin{center}
    \begin{figure}
      \begin{tikzpicture}[node distance=2cm, scale=3]
        \node (10) at (1,0)  {$\Luv$};
        \node (21) at (2,1)  {$K$};
        \node (01) at (0,1)  {$L$};
        \node (middle) at (1,0.6)  {$\circlearrowleft$};
        \draw[->,thick] (01) -- (21)  node[pos=.5,above] {$\phi$};
        \draw[->,thick] (01) -- (10)  node[pos=.5,left] {$\suv$};
        \draw[->,dashed,thick] (10) -- (21)  node[pos=.5,right] {$\exists \,\psi$};
      \end{tikzpicture}
      \caption{Every join-homomorphism collapsing $u$ and $v$ is divisible by $\suv$.}
      \label{fig:splitting}
    \end{figure}
  \end{center}

  
\end{example}


\subsection{Algebraicity}
A subset $D$ of an ordered set $P$ is called \defn{up-directed}
if for every $x, y \in D$ there exists $z \in D$ such that
$x \leq z$ and $y \leq z$. 
A closure operator $\cl\colon \sP(X) \to \sP(X)$ is called \defn{algebraic} provided,
for all $A\subseteq X$, 
\[
\cl A = \bigcup \{\cl F \mid F \subseteq A \text{ and } F \text{ finite } \}.
\]
The collection $\sC$ of closed sets of an algebraic closure operator is called an
\defn{algebraic closure system}.
\begin{theorem}[cf.~\cite{Nation-notes} Thm 3.1]
  Let $\sC$ be the closure system of fixed points of the closure operator $\cl\colon \sP(X) \to \sP(X)$.
  The following are equivalent:
\begin{enumerate}
\item $\cl$ is an algebraic closure operator
\item $\sC$ is an algebraic closure system
\item If $D\subseteq \sC$ is up-directed and $C\subseteq D$ is a chain, then $\bigcup C$ is closed.
\item If $D\subseteq \sC$ is up-directed, then $\bigcup D$ is closed.
\item If $C\subseteq \sC$ is a chain, then $\bigcup C$ is closed.
\end{enumerate}
\end{theorem}

For an algebra $\bA$, the subalgebra generation operator $\Sg^{\bA}$ is an algebraic
closure operator (on the poset $\<\sP(A), \subseteq\>$) whose fixpoints are subalgebras of $\bA$.
Thus the lattice $\<\Sub(\bA), \join, \meet\>$ of subalgebras of $\bA$ is an algebraic closure
system.
Conversely, given an algebraic closure system $\sS$, we can construct an algebra $\bA =\<A, F\>$ so that
$\sS = \Sub(\bA)$.
Let $\cl_{\sS}$ denote the corresponding closure operator.
Let $a_0, a_1, \dots, a_{n-1}, b \in A$ be such that
$b \in \cl_{\sS} \{a_0, a_1, \dots, a_{n-1}\}$.  Define an $n$-ary operation
$f_{\ba, b}$ so that $f_{\ba, b}(a_0, a_1, \dots, a_{n-1}) = b$ and
for all other tuples $f_{\ba, b}(c_0, c_1, \dots, c_{n-1}) = c_0$.
Do this for each $\ba = (a_0, a_1, \dots, a_{n-1})$. Then define
\[
F = \{f_{\ba, b} \mid  n< \omega, \, \ba \in A^n,  \, b \in \cl_{\sS} \ba\}.
\]
Every subalgebra is the union of its finitely generated subalgebras.
%% \appendix
%% \section{Appendix Title}
%% This is the text of the appendix, if you need one.

%\bibliographystyle{amsplain} %% or amsalpha
%% \bibliographystyle{plain-url}
\printbibliography


\end{document}





















